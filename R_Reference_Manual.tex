\PassOptionsToPackage{unicode=true}{hyperref} % options for packages loaded elsewhere
\PassOptionsToPackage{hyphens}{url}
%
\documentclass[]{book}
\usepackage{lmodern}
\usepackage{amssymb,amsmath}
\usepackage{ifxetex,ifluatex}
\usepackage{fixltx2e} % provides \textsubscript
\ifnum 0\ifxetex 1\fi\ifluatex 1\fi=0 % if pdftex
  \usepackage[T1]{fontenc}
  \usepackage[utf8]{inputenc}
  \usepackage{textcomp} % provides euro and other symbols
\else % if luatex or xelatex
  \usepackage{unicode-math}
  \defaultfontfeatures{Ligatures=TeX,Scale=MatchLowercase}
\fi
% use upquote if available, for straight quotes in verbatim environments
\IfFileExists{upquote.sty}{\usepackage{upquote}}{}
% use microtype if available
\IfFileExists{microtype.sty}{%
\usepackage[]{microtype}
\UseMicrotypeSet[protrusion]{basicmath} % disable protrusion for tt fonts
}{}
\IfFileExists{parskip.sty}{%
\usepackage{parskip}
}{% else
\setlength{\parindent}{0pt}
\setlength{\parskip}{6pt plus 2pt minus 1pt}
}
\usepackage{hyperref}
\hypersetup{
            pdftitle={R Reference Manual},
            pdfborder={0 0 0},
            breaklinks=true}
\urlstyle{same}  % don't use monospace font for urls
\usepackage{color}
\usepackage{fancyvrb}
\newcommand{\VerbBar}{|}
\newcommand{\VERB}{\Verb[commandchars=\\\{\}]}
\DefineVerbatimEnvironment{Highlighting}{Verbatim}{commandchars=\\\{\}}
% Add ',fontsize=\small' for more characters per line
\usepackage{framed}
\definecolor{shadecolor}{RGB}{248,248,248}
\newenvironment{Shaded}{\begin{snugshade}}{\end{snugshade}}
\newcommand{\AlertTok}[1]{\textcolor[rgb]{0.94,0.16,0.16}{#1}}
\newcommand{\AnnotationTok}[1]{\textcolor[rgb]{0.56,0.35,0.01}{\textbf{\textit{#1}}}}
\newcommand{\AttributeTok}[1]{\textcolor[rgb]{0.77,0.63,0.00}{#1}}
\newcommand{\BaseNTok}[1]{\textcolor[rgb]{0.00,0.00,0.81}{#1}}
\newcommand{\BuiltInTok}[1]{#1}
\newcommand{\CharTok}[1]{\textcolor[rgb]{0.31,0.60,0.02}{#1}}
\newcommand{\CommentTok}[1]{\textcolor[rgb]{0.56,0.35,0.01}{\textit{#1}}}
\newcommand{\CommentVarTok}[1]{\textcolor[rgb]{0.56,0.35,0.01}{\textbf{\textit{#1}}}}
\newcommand{\ConstantTok}[1]{\textcolor[rgb]{0.00,0.00,0.00}{#1}}
\newcommand{\ControlFlowTok}[1]{\textcolor[rgb]{0.13,0.29,0.53}{\textbf{#1}}}
\newcommand{\DataTypeTok}[1]{\textcolor[rgb]{0.13,0.29,0.53}{#1}}
\newcommand{\DecValTok}[1]{\textcolor[rgb]{0.00,0.00,0.81}{#1}}
\newcommand{\DocumentationTok}[1]{\textcolor[rgb]{0.56,0.35,0.01}{\textbf{\textit{#1}}}}
\newcommand{\ErrorTok}[1]{\textcolor[rgb]{0.64,0.00,0.00}{\textbf{#1}}}
\newcommand{\ExtensionTok}[1]{#1}
\newcommand{\FloatTok}[1]{\textcolor[rgb]{0.00,0.00,0.81}{#1}}
\newcommand{\FunctionTok}[1]{\textcolor[rgb]{0.00,0.00,0.00}{#1}}
\newcommand{\ImportTok}[1]{#1}
\newcommand{\InformationTok}[1]{\textcolor[rgb]{0.56,0.35,0.01}{\textbf{\textit{#1}}}}
\newcommand{\KeywordTok}[1]{\textcolor[rgb]{0.13,0.29,0.53}{\textbf{#1}}}
\newcommand{\NormalTok}[1]{#1}
\newcommand{\OperatorTok}[1]{\textcolor[rgb]{0.81,0.36,0.00}{\textbf{#1}}}
\newcommand{\OtherTok}[1]{\textcolor[rgb]{0.56,0.35,0.01}{#1}}
\newcommand{\PreprocessorTok}[1]{\textcolor[rgb]{0.56,0.35,0.01}{\textit{#1}}}
\newcommand{\RegionMarkerTok}[1]{#1}
\newcommand{\SpecialCharTok}[1]{\textcolor[rgb]{0.00,0.00,0.00}{#1}}
\newcommand{\SpecialStringTok}[1]{\textcolor[rgb]{0.31,0.60,0.02}{#1}}
\newcommand{\StringTok}[1]{\textcolor[rgb]{0.31,0.60,0.02}{#1}}
\newcommand{\VariableTok}[1]{\textcolor[rgb]{0.00,0.00,0.00}{#1}}
\newcommand{\VerbatimStringTok}[1]{\textcolor[rgb]{0.31,0.60,0.02}{#1}}
\newcommand{\WarningTok}[1]{\textcolor[rgb]{0.56,0.35,0.01}{\textbf{\textit{#1}}}}
\usepackage{longtable,booktabs}
% Fix footnotes in tables (requires footnote package)
\IfFileExists{footnote.sty}{\usepackage{footnote}\makesavenoteenv{longtable}}{}
\usepackage{graphicx,grffile}
\makeatletter
\def\maxwidth{\ifdim\Gin@nat@width>\linewidth\linewidth\else\Gin@nat@width\fi}
\def\maxheight{\ifdim\Gin@nat@height>\textheight\textheight\else\Gin@nat@height\fi}
\makeatother
% Scale images if necessary, so that they will not overflow the page
% margins by default, and it is still possible to overwrite the defaults
% using explicit options in \includegraphics[width, height, ...]{}
\setkeys{Gin}{width=\maxwidth,height=\maxheight,keepaspectratio}
\setlength{\emergencystretch}{3em}  % prevent overfull lines
\providecommand{\tightlist}{%
  \setlength{\itemsep}{0pt}\setlength{\parskip}{0pt}}
\setcounter{secnumdepth}{5}
% Redefines (sub)paragraphs to behave more like sections
\ifx\paragraph\undefined\else
\let\oldparagraph\paragraph
\renewcommand{\paragraph}[1]{\oldparagraph{#1}\mbox{}}
\fi
\ifx\subparagraph\undefined\else
\let\oldsubparagraph\subparagraph
\renewcommand{\subparagraph}[1]{\oldsubparagraph{#1}\mbox{}}
\fi

% set default figure placement to htbp
\makeatletter
\def\fps@figure{htbp}
\makeatother

\usepackage{booktabs}
\usepackage[]{natbib}
\bibliographystyle{apalike}

\title{R Reference Manual}
\author{}
\date{\vspace{-2.5em}}

\begin{document}
\maketitle

{
\setcounter{tocdepth}{1}
\tableofcontents
}
\hypertarget{introduction}{%
\chapter{Introduction}\label{introduction}}

I organized this manual according to Garrett Grolemund and Handley Wickham's conception of the tools needed to tackle about 80\% of the tasks required in a typical data science project (``\href{https://r4ds.had.co.nz/introduction.html}{Introduction}'', \emph{R for Data Science}). Those tools are:

\begin{itemize}
\tightlist
\item
  Import
\item
  Tidy
\item
  Transform
\item
  Visualize
\item
  Model
\item
  Communicate
\item
  Program.

  \hypertarget{htmlwidget-afbf728bf59138443395}{}
\end{itemize}

\hypertarget{import-or-create-data}{%
\chapter{Import or Create Data}\label{import-or-create-data}}

\begin{center}\rule{0.5\linewidth}{\linethickness}\end{center}

\hypertarget{notes-references}{%
\section{Notes \& References}\label{notes-references}}

\begin{itemize}
\tightlist
\item
  \href{http://adv-r.had.co.nz/Data-structures.html}{``Data structures''} (Wickham, \href{http://adv-r.had.co.nz/}{\emph{Advanced R}})
\item
  \texttt{\{open\_intro\}} contains data sets useful for practicing and teaching.
\item
  \href{https://r4ds.had.co.nz/vectors.html}{``Vectors''} (Grolemund \& Wickham, \href{https://r4ds.had.co.nz/}{\emph{R for Data Science}})
\end{itemize}

\hypertarget{create-data}{%
\section{Create Data}\label{create-data}}

\begin{itemize}
\tightlist
\item
  \texttt{base::array}
\item
  \texttt{base::c}
\item
  \texttt{base::data.frame}
\item
  \texttt{base::factor}
\item
  \texttt{base::list}
\item
  \texttt{base::matrix}
\item
  \texttt{base::seq}
\item
  \texttt{base::vector}

  \begin{itemize}
  \tightlist
  \item
    Preferable to \texttt{base::c} when creating an empty vector, according to \href{https://www.datacamp.com/community/tutorials/five-tips-r-code-improve}{``Five Tips to Improve Your R Code''}.
  \end{itemize}
\item
  \texttt{stats::rnorm}
\item
  \texttt{tibble::add\_row}
\item
  \texttt{tibble::tibble}
\end{itemize}

\hypertarget{import-data-from-a-local-drive}{%
\section{Import Data from a Local Drive}\label{import-data-from-a-local-drive}}

\texttt{\{base\}}

\begin{itemize}
\tightlist
\item
  \texttt{attach}: Attach a set of R objects to the search path.

  \begin{itemize}
  \tightlist
  \item
    Allows objects in the database to be accessed by giving their names (e.g., \texttt{height} rather than \texttt{women\$height}).
  \end{itemize}
\item
  \texttt{file.choose}: Choose a file interactively.

  \begin{itemize}
  \tightlist
  \item
    Use as \texttt{file\ =\ file.choose()} inside a data import function (such as \texttt{read\_csv} and \texttt{readRDS}).
  \end{itemize}
\item
  \texttt{load}: Reload datasets saved with \texttt{save()}.
\item
  \texttt{readRDS}: Restore an R object written with \texttt{saveRDS()}.
\end{itemize}

\texttt{\{data.table\}}

\begin{itemize}
\tightlist
\item
  \texttt{fread}: Similar to \texttt{read.table}, but faster and more convenient for large data sets.
\end{itemize}

\texttt{\{foreign\}}

\begin{itemize}
\tightlist
\item
  \texttt{read.spss}: Read an SPSS data file.
\end{itemize}

\texttt{\{haven\}}

\begin{itemize}
\tightlist
\item
  \texttt{read\_sas}: Read and write SAS files.
\end{itemize}

\texttt{\{readr\}}

\begin{itemize}
\tightlist
\item
  \texttt{read\_csv}: Read a comma separated value file into a tibble.
\item
  \texttt{read\_csv2}: Read a semi-colon separated value file into a tibble.
\item
  \texttt{read\_delim}: Read a delimited file into a tibble.
\item
  \texttt{read\_tsv}: Read a tab separated value file into a tibble.
\end{itemize}

\texttt{\{readxl\}}

\begin{itemize}
\tightlist
\item
  \texttt{excel\_sheets}: List all sheets in an Excel spreadsheet.
\item
  \texttt{read\_excel}: Read xls and xlsx files.
\item
  \texttt{read\_xls}: Read a .xls file.
\item
  \texttt{read\_xlsx}: Read a .xlsx file.
\end{itemize}

\texttt{\{utils\}}

\begin{itemize}
\tightlist
\item
  \texttt{data}: Load specified data sets, or list the available data sets.

  \begin{itemize}
  \tightlist
  \item
    Use this function to load the data sets that accompany R packages, such as \texttt{openintro}'s \texttt{hsb2} and \texttt{email50} and \texttt{gapminder}'s \texttt{gapminder}.
  \end{itemize}
\item
  \texttt{read.csv}: Read a .csv file.
\item
  \texttt{read.csv2}: Read semi-colon separated value file.
\item
  \texttt{read.delim}: Read a delimited file.
\item
  \texttt{read.delim2}: Read a delimited file where the decimal point is a comma.
\item
  \texttt{read.table}: Read a file in table format.
\end{itemize}

\texttt{\{XLConnect\}}

\begin{itemize}
\tightlist
\item
  \texttt{readWorksheetFromFile}: Read data from worksheets in an Excel file.
\end{itemize}

\hypertarget{import-data-from-the-internet}{%
\section{Import Data from the Internet}\label{import-data-from-the-internet}}

\texttt{\{httr\}}

\begin{itemize}
\tightlist
\item
  \texttt{GET}: Get a URL.
\end{itemize}

\texttt{\{jsonlite\}}

\begin{itemize}
\tightlist
\item
  \texttt{read\_json}: Read and write JSON.
\end{itemize}

\texttt{\{readr\}}

\begin{itemize}
\tightlist
\item
  \texttt{read\_csv}: Read a comma separated value file into a tibble.
\item
  \texttt{read\_csv2}: Read a semi-colon separated value file into a tibble.
\item
  \texttt{read\_delim}: Read a delimited file into a tibble.
\item
  \texttt{read\_tsv}: Read a tab separated value file into a tibble.
\end{itemize}

\texttt{\{rjson\}}

\begin{itemize}
\tightlist
\item
  \texttt{fromJSON}: Convert JSON to R.
\end{itemize}

\texttt{\{utils\}}

\begin{itemize}
\tightlist
\item
  \texttt{download.file}: Download a file from the Internet.

  \begin{itemize}
  \tightlist
  \item
    Example:
  \end{itemize}
\end{itemize}

\begin{Shaded}
\begin{Highlighting}[]
\KeywordTok{download.file}\NormalTok{(}
  \StringTok{"https://assets.datacamp.com/production/repositories/5028/datasets/a55843f83746968c7f118d82ed727db9c71e891f/snake_river_visits.rds"}\NormalTok{,}
  \DataTypeTok{destfile =} \KeywordTok{paste0}\NormalTok{(}\KeywordTok{getwd}\NormalTok{(), }\StringTok{"/Snake River Visits.rds"}\NormalTok{))}

\NormalTok{snake_river_visits <-}\StringTok{ }\KeywordTok{readRDS}\NormalTok{(}\KeywordTok{file.choose}\NormalTok{())}
\CommentTok{# or}
\NormalTok{path <-}\StringTok{ }\KeywordTok{paste0}\NormalTok{(}\KeywordTok{getwd}\NormalTok{(), }\StringTok{"/Snake River Visits.rds"}\NormalTok{)}
\NormalTok{snake_river_visits <-}\StringTok{ }\KeywordTok{readRDS}\NormalTok{(path)}
\end{Highlighting}
\end{Shaded}

\begin{itemize}
\tightlist
\item
  Example: Rdata files
\end{itemize}

\begin{Shaded}
\begin{Highlighting}[]
\CommentTok{# Example 1:}
\KeywordTok{download.file}\NormalTok{(}
  \StringTok{"https://assets.datacamp.com/production/repositories/236/datasets/7f714f993f1ad4c3d26412ae1e537ce6355b1b54/iris.RData"}\NormalTok{, }
  \DataTypeTok{destfile =} \StringTok{"datacamp_iris_dataset.Rdata"}\NormalTok{)}

\KeywordTok{load}\NormalTok{(}\StringTok{"datacamp_iris_dataset.Rdata"}\NormalTok{)}

\CommentTok{# Example 2:}
\KeywordTok{download.file}\NormalTok{(}
  \StringTok{"https://assets.datacamp.com/production/repositories/235/datasets/3b6fc2923b599058584b57d8c605c6bef454d273/CHIS2009_reduced_2.Rdata"}\NormalTok{,}
  \DataTypeTok{destfile =} \StringTok{"chis_2009.Rdata"}\NormalTok{,}
  \CommentTok{# The documentation for `download.file` indicates that the function will}
  \CommentTok{# automatically include `mode = "wb"` for .Rdata files. That may have happened}
  \CommentTok{# in Example 1, but didn't happen in Example 2, which is why I've included it.}
  \DataTypeTok{mode =} \StringTok{"wb"}\NormalTok{)}

\KeywordTok{load}\NormalTok{(}\StringTok{"chis_2009.Rdata"}\NormalTok{)}
\end{Highlighting}
\end{Shaded}

\begin{itemize}
\tightlist
\item
  \texttt{unzip}: Extract or list zip archives.

  \begin{itemize}
  \tightlist
  \item
    Example:
  \end{itemize}
\end{itemize}

\begin{Shaded}
\begin{Highlighting}[]
\KeywordTok{download.file}\NormalTok{(}
  \StringTok{"https://assets.datacamp.com/production/repositories/1069/datasets/578834f5908e3b2fa575429a287586d1eaeb2e54/countries2.zip"}\NormalTok{,}
  \DataTypeTok{destfile =} \StringTok{"Data Sets/Countries"}\NormalTok{,}
  \DataTypeTok{mode =} \StringTok{"wb"}\NormalTok{)}

\KeywordTok{unzip}\NormalTok{(}\StringTok{"Data Sets/Countries"}\NormalTok{, }\DataTypeTok{exdir =} \StringTok{"Data Sets"}\NormalTok{)}
\end{Highlighting}
\end{Shaded}

\hypertarget{import-data-from-a-database}{%
\section{Import Data from a Database}\label{import-data-from-a-database}}

\texttt{DBI}

\begin{itemize}
\tightlist
\item
  \texttt{dbBind()}: Bind values to a parameterized/prepared statement.
\item
  \texttt{dbClearResult()}: Free all resources (local and remote) associated with a result set.
\item
  \texttt{dbConnect()}: Connect to a DBMS.
\item
  \texttt{dbDataType()}: Determine the SQL data type of an object.
\item
  \texttt{dbDisconnect()}: Disconnect (close) a connection to a DBMS.
\item
  \texttt{dbFetch()}: Fetch records from a previously executed query.
\item
  \texttt{dbGetQuery()}: Send query, retrieve the results, and then clear result set.
\item
  \texttt{dbListTables()}: List remote tables.
\item
  \texttt{dbReadTable()}: Copy data frames to and from database tables.
\item
  \texttt{dbSendQuery()}: Execute a query on a given database connection.
\item
  \texttt{dbSendStatement()}: Execute a data manipulation statement on a given database connection.
\end{itemize}

\hypertarget{reference-material}{%
\section{Reference Material}\label{reference-material}}

The \texttt{openintro} package contains data sets useful for practicing and teaching.

\hypertarget{tidy}{%
\chapter{Tidy}\label{tidy}}

\begin{center}\rule{0.5\linewidth}{\linethickness}\end{center}

``Tidying your data means storing it in a consistent form that matches the semantics of the dataset with the way it is stored. In brief, when your data is tidy, each column is a variable and each row is an observation. Tidying data is important because the consistent structure lets you focus your struggle on questions about the data, not fighting to get the data into the right form for different functions.''\\
- Garrett Grolemund \& Hadley Wickham, \emph{R for Data Science}

\begin{center}\rule{0.5\linewidth}{\linethickness}\end{center}

\hypertarget{explore-raw-data}{%
\section{Explore Raw Data}\label{explore-raw-data}}

\hypertarget{understand-the-structure-of-the-data}{%
\subsection{Understand the Structure of the Data}\label{understand-the-structure-of-the-data}}

\texttt{\{base\}}

\begin{itemize}
\tightlist
\item
  \texttt{attr}: Get or set attributes of an object.

  \begin{itemize}
  \tightlist
  \item
    \texttt{attr(x,\ "names")} is the same as \texttt{names(x)}.
  \item
    Use \texttt{attr(x,\ "names")\ \textless{}-\ value} to set attribute values.
  \end{itemize}
\item
  \texttt{attributes}: Object attribute lists.
\item
  \texttt{class}: Get or set the class attribute of an object.
\item
  \texttt{colnames}: Retrieve or set the column names of a matrix-like object.
\item
  \texttt{dim}: Retrieve or set the dimension of an object.
\item
  \texttt{dimnames}: Retrieve or set the dimension names of an object.
\item
  \texttt{format}: Format an R object.
\item
  \texttt{length}: Get or set the length of an object.
\item
  \texttt{levels}: Get or set the (factor) levels of a variable.

  \begin{itemize}
  \tightlist
  \item
    Levels default to alphabetical order, so be careful when renaming them (i.e., don't accidentally set the ``F'' level equal to ``Male'' rather than ``Female'').
  \end{itemize}
\item
  \texttt{mode}: Get or set the storage mode attribute of an object.

  \begin{itemize}
  \tightlist
  \item
    Modes include logical, numeric (the mode equivalent of \texttt{typeof}'s integer and double types), complex, character, raw, and list.
  \end{itemize}
\item
  \texttt{names}: Get or set the names of an object.
\item
  \texttt{nchar}: Count the number of characters (or bytes or width).
\item
  \texttt{rownames}: Retrieve or set the row names of a matrix-like object.
\item
  \texttt{typeof}: Display the R internal type of an object.

  \begin{itemize}
  \tightlist
  \item
    Types include logical, integer, double, complex, character, raw, and list.
  \end{itemize}
\end{itemize}

\texttt{\{tibble\}}

\begin{itemize}
\tightlist
\item
  \texttt{glimpse}: Get a glimpse of the data.

  \begin{itemize}
  \tightlist
  \item
    Similar to \texttt{utils::str}.
  \end{itemize}
\end{itemize}

\texttt{\{utils\}}

\begin{itemize}
\tightlist
\item
  \texttt{str()}: Display the structure of an R object.

  \begin{itemize}
  \tightlist
  \item
    Similar to \texttt{tibble::glimpse}.
  \end{itemize}
\end{itemize}

\hypertarget{look-at-the-data}{%
\subsection{Look at the Data}\label{look-at-the-data}}

\texttt{base}

\begin{itemize}
\tightlist
\item
  \texttt{names()}: Get or set the names of an object.
\item
  \texttt{order()}: Rearrange in ascending or descending order.
\item
  \texttt{summary()}: Summarize the object.
\end{itemize}

\texttt{utils}

\begin{itemize}
\tightlist
\item
  \texttt{head()}: View the first observations in a data frame.
\item
  \texttt{tail()}: View the last observations in a data frame.
\end{itemize}

\hypertarget{visualize-the-data}{%
\subsection{Visualize the Data}\label{visualize-the-data}}

\texttt{graphics}

\begin{itemize}
\tightlist
\item
  \texttt{hist()}: Create a histogram.
\item
  \texttt{plot()}: Create an x-y plot.
\end{itemize}

\begin{center}\rule{0.5\linewidth}{\linethickness}\end{center}

\hypertarget{tidy-data}{%
\section{Tidy Data}\label{tidy-data}}

\hypertarget{manage-columns-and-observations}{%
\subsection{Manage Columns and Observations}\label{manage-columns-and-observations}}

\texttt{\{base\}}

\begin{itemize}
\tightlist
\item
  \texttt{duplicated}: Determine duplicate elements.

  \begin{itemize}
  \tightlist
  \item
    See also \texttt{data.table::duplicated}.
  \end{itemize}
\item
  \texttt{unique}: Extract unique elements.

  \begin{itemize}
  \tightlist
  \item
    See also \texttt{data.table::unique}.
  \end{itemize}
\end{itemize}

\texttt{\{data.table\}}

\begin{itemize}
\tightlist
\item
  \texttt{anyDuplicated}: Indicate the index of the first duplicate entry.
\item
  \texttt{duplicated}: Return a logical vector indicating whether a row is a duplicate.

  \begin{itemize}
  \tightlist
  \item
    See also \texttt{base::duplicated}.
  \end{itemize}
\item
  \texttt{unique}: Remove duplicate rows.

  \begin{itemize}
  \tightlist
  \item
    See also \texttt{base::unique}.
  \end{itemize}
\end{itemize}

\texttt{\{janitor\}}

\begin{itemize}
\tightlist
\item
  \texttt{remove\_empty}: Remove empty rows and/or columns from a data.frame or matrix.
\end{itemize}

\texttt{\{splitstackshape\}}

\begin{itemize}
\tightlist
\item
  \texttt{cSplit}: Split concatencated values into separate values.
\end{itemize}

\texttt{\{tibble\}}

\begin{itemize}
\tightlist
\item
  \texttt{rownames}: Tools for working with row names.

  \begin{itemize}
  \tightlist
  \item
    \texttt{rowid\_to\_column}: Add a column of sequential row IDs.

    \begin{itemize}
    \tightlist
    \item
      Useful when a separate ID is required to manipulate rows or columns, such as when using \texttt{tidyr::gather}.
    \end{itemize}
  \end{itemize}
\end{itemize}

\texttt{\{tidyr\}}

\begin{itemize}
\tightlist
\item
  \texttt{gather}: Gather columns into key-value pairs.
\item
  \texttt{nest}: Nest repeated values in a list-variable.

  \begin{itemize}
  \tightlist
  \item
    Helpful when separating a data frame in preparation to model the data for each grouping.
  \end{itemize}
\item
  \texttt{replace\_na}: Replace missing values.
\item
  \texttt{separate}: Separate one column into multiple columns.
\item
  \texttt{spread}: Spread across multiple columns.
\item
  \texttt{unite}: Unite multiple columns into one.
\item
  \texttt{unnest}: Unnest a list-column.
\end{itemize}

\hypertarget{transpose}{%
\subsection{Transpose}\label{transpose}}

\texttt{\{purrr\}}

\begin{itemize}
\tightlist
\item
  \texttt{transpose}: Turn a list-of-lists inside-out.
\end{itemize}

\begin{center}\rule{0.5\linewidth}{\linethickness}\end{center}

\hypertarget{prepare-data-for-analysis}{%
\section{Prepare Data for Analysis}\label{prepare-data-for-analysis}}

\hypertarget{coerce-data}{%
\subsection{Coerce Data}\label{coerce-data}}

\texttt{\{base\}}

\begin{itemize}
\tightlist
\item
  \texttt{anyNA}: Possibly faster implementation of \texttt{any(is.na(x))}.
\item
  \texttt{as.*}

  \begin{itemize}
  \tightlist
  \item
    \texttt{as.array}: Coerce to array.
  \item
    \texttt{as.data.frame}: Coerce to data frame.

    \begin{itemize}
    \tightlist
    \item
      Prefer\texttt{tibble::as\_tibble} to \texttt{base::as.data.frame}.
    \end{itemize}
  \item
    \texttt{as.Date}: Coerce to date.
  \item
    \texttt{as.factor}: Coerce to factor.
  \item
    \texttt{as.list}: Coerce to list.
  \item
    \texttt{as.matrix}: Coerce to matrix.
  \item
    \texttt{as.POSIX*}: Coerce to POSIXlt or POSIXct.
  \end{itemize}
\item
  \texttt{is.na}: Indicate which elements are missing.

  \begin{itemize}
  \tightlist
  \item
    Use \texttt{is.na\ \textless{}-} to set elements to \texttt{NA}.
  \end{itemize}
\item
  \texttt{unclass}: Remove the class attribute of an object.
\end{itemize}

\texttt{\{methods\}}

\begin{itemize}
\tightlist
\item
  \texttt{as}: Force an object to belong to a class.
\end{itemize}

\texttt{\{tibble\}}

\begin{itemize}
\tightlist
\item
  \texttt{as\_tibble}: Coerce lists and matrices to data frames.

  \begin{itemize}
  \tightlist
  \item
    Preferable to \texttt{base::as.data.frame}.
  \end{itemize}
\item
  \texttt{enframe}: Convert vectors to data frames, and vice versa.

  \begin{itemize}
  \tightlist
  \item
    Preferable to using \texttt{as\_tibble} to coerce a vector to a data frame.
  \end{itemize}
\end{itemize}

\hypertarget{dates-and-datetimes}{%
\subsection{Dates and Datetimes}\label{dates-and-datetimes}}

\texttt{anytime}

\begin{itemize}
\tightlist
\item
  \texttt{anytime()}: Parse POSIXct or Date objects from input data.
\end{itemize}

\texttt{base}

\begin{itemize}
\tightlist
\item
  \texttt{as.Date()}: Date conversion to and from character.
\item
  \texttt{as.POSIX*()}: Date-time conversion for POSIXct and POSIXlt.

  \begin{itemize}
  \tightlist
  \item
    \texttt{as.POSIXct()}: Setting default for UTC and 1970.
  \end{itemize}
\item
  \texttt{strptime()}: Date-time conversion to and from character.
\item
  \texttt{Sys.timezone()}: Return the name of the current time zone.

  \begin{itemize}
  \tightlist
  \item
    \texttt{OlsonNames()} displays available time zones.
  \end{itemize}
\end{itemize}

\texttt{fasttime}

\begin{itemize}
\tightlist
\item
  \texttt{fastPOSIXct()}: Convert strings into POSICct object (string must be in year, month, day, hour, minute, second format.)
\end{itemize}

\texttt{hms}

\begin{itemize}
\tightlist
\item
  \texttt{hms()}: Store time-of-day values as \texttt{hms} class.

  \begin{itemize}
  \tightlist
  \item
    Child functions: \texttt{as.hms()}, `is.hms().
  \end{itemize}
\end{itemize}

\texttt{lubridate}

\begin{itemize}
\tightlist
\item
  \texttt{as\_date()}: Convert an object to a date or date-time.
\item
  \texttt{parse\_date\_time()}: User friendly date-time parsing functions that can accomodate parsing multiple dates in different formats.

  \begin{itemize}
  \tightlist
  \item
    \texttt{fast\_strptime()}: Fast C parser of numeric formats only that accepts explicit format arguments, just as \texttt{base::strptime()}.

    \begin{itemize}
    \tightlist
    \item
      Note that the format argument must match the input exactly, including any non-white space characters (such as ``T'' and ``Z'').
    \end{itemize}
  \item
    \texttt{make\_date()}: Create dates from numeric representations.
  \item
    \texttt{make\_datetime()}: Create date-times from numeric representations.
  \item
    \texttt{parse\_date\_time2()}: Fast C parser of numeric orders.
  \item
    \texttt{parse\_date\_time()} can be slow because it is designed to be forgiving and flexible. If the dates you are working with are in a consistent format (ideally ISO 8601), use one of the following: \texttt{fasttime::fastPOSIXct()}
  \end{itemize}
\item
  \texttt{ymd()}: Parse dates with year, month, and day components.
  + Related formats: \texttt{ydm()}, \texttt{mdy()}, \texttt{myd()}, \texttt{dmy()}, \texttt{dym()}, \texttt{yq()}.
\item
  \texttt{ymd\_hms()}: Parse date-times with year, month, day, hour, minute, and second components.
  + Related formats: \texttt{ymd\_hm()}, \texttt{ymd\_h()}, \texttt{dmy\_hms()}, \texttt{dmy\_hm()}, \texttt{dmy\_h()}, \texttt{mdy\_hms()}, \texttt{mdy\_hm()}, \texttt{mdy\_h()}, \texttt{ydm\_hms()}, \texttt{ydm\_hm()}, \texttt{ydm\_h()}.
\end{itemize}

\hypertarget{factors-and-levels}{%
\subsection{Factors and Levels}\label{factors-and-levels}}

\texttt{base}

\begin{itemize}
\tightlist
\item
  \texttt{factor()}: Get and set factors.

  \begin{itemize}
  \tightlist
  \item
    Rearrange the order of factors by using the \texttt{levels} argument. For example, rearrange the order of ``Bad,''Good," and ``Neutral'' using `levels = c(``Bad'', ``Neutral'', ``Good'').
  \end{itemize}
\end{itemize}

\begin{center}\rule{0.5\linewidth}{\linethickness}\end{center}

\hypertarget{filter-and-remove-data}{%
\subsection{Filter and Remove Data}\label{filter-and-remove-data}}

\texttt{purrr}

\begin{itemize}
\tightlist
\item
  \texttt{keep()}: Keep or discard elements using a predicate function.
\end{itemize}

\texttt{stats}

\begin{itemize}
\tightlist
\item
  \texttt{na.omit()}: Remove rows with \texttt{NA} values.
\end{itemize}

\hypertarget{strings}{%
\subsection{Strings}\label{strings}}

\texttt{base}

\begin{itemize}
\tightlist
\item
  \texttt{cat}: Concatenate and print.
\item
  \texttt{chartr}: Change certain characters.
\item
  \texttt{gregexpr}:
\item
  \texttt{grep}: Pattern matching and replacement.
\item
  \texttt{grepl}:
\item
  \texttt{gsub}:
\item
  \texttt{regexec}:
\item
  \texttt{regexpr}:
\item
  \texttt{sub}:
\item
  \texttt{tolower}: Convert to lowercase.

  \begin{itemize}
  \tightlist
  \item
    \texttt{stringr::str\_to\_lower} is an alternative.
  \end{itemize}
\item
  \texttt{toupper}: Convert to uppercase.

  \begin{itemize}
  \tightlist
  \item
    \texttt{stringr::str\_to\_upper} is an alternative.
  \end{itemize}
\end{itemize}

\texttt{qdap}

\begin{itemize}
\tightlist
\item
  \texttt{check\_spelling}
\end{itemize}

\texttt{qdapDictionaries}

\begin{itemize}
\tightlist
\item
  \texttt{DICTIONARY}: Nettalk Corpus syllable data set.
\item
  \texttt{GradyAugmented}: Augmented list of Grady Ward's \emph{English Words} and Mark Kantrowitz's \emph{Names} List.

  \begin{itemize}
  \tightlist
  \item
    Mark Kantrowitz's \emph{Names} list is available in full \href{http://www.cs.cmu.edu/afs/cs/project/ai-repository/ai/areas/nlp/corpora/names/}{here}.
  \end{itemize}
\end{itemize}

\texttt{stringr}

\begin{itemize}
\tightlist
\item
  \texttt{str\_detect}: Detect the presence or absence of a pattern in a string.

  \begin{itemize}
  \tightlist
  \item
    Control the \texttt{pattern} argument options with \texttt{regex} (e.g., \texttt{str\_detect(x,\ regex(pattern,\ ignore\_case\ =\ TRUE))}.
  \end{itemize}
\item
  \texttt{str\_remove}: Remove matched patterns in a string.
\item
  \texttt{str\_to\_lower}: Convert to lower case.
\item
  \texttt{str\_to\_title}: Capitalize the first letter.
\item
  \texttt{str\_trim}: Trim whitespace from a string.
\item
  \texttt{str\_to\_upper}: Convert to upper case.
\end{itemize}

\hypertarget{test-data}{%
\subsection{Test Data}\label{test-data}}

\texttt{base}

\begin{itemize}
\tightlist
\item
  \texttt{all()}: Are all values true?
\item
  \texttt{any()}: Are any values true?

  \begin{itemize}
  \tightlist
  \item
    Use \texttt{any(is.na(data.frame))} to determine if there are any NA values in a data frame.
  \end{itemize}
\item
  \texttt{exists()}: Check whether an R object exists.
\item
  \texttt{is.*()} functions:

  \begin{itemize}
  \tightlist
  \item
    \texttt{is.array()}: Test whether an object is an array.
  \item
    \texttt{is.data.frame()}: Test whether an object is a data frame.
  \item
    \texttt{is.matrix()}: Test whether an object is a matrix.
  \item
    \texttt{is.vector()}: Test whether an object is a vector.
  \end{itemize}
\item
  \texttt{setequal()}: Check two vectors for equality.
\item
  \texttt{sum()}: Sum vector elements.

  \begin{itemize}
  \tightlist
  \item
    To test whether all elements of a vector do or do not meet a certain condition, use as follows: \texttt{sum(email\$num\_char\ \textless{}\ 0)}.
  \end{itemize}
\end{itemize}

\texttt{purrr}

\begin{itemize}
\tightlist
\item
  \texttt{every()}: Do every or some elements of a list satisfy a predicate?
\end{itemize}

\texttt{stats}

\begin{itemize}
\tightlist
\item
  \texttt{complete.cases()}: Find complete cases (i.e., rows without \texttt{NA} values).
\end{itemize}

\texttt{tibble}

\begin{itemize}
\tightlist
\item
  \texttt{is\_tibble()}: Test whether an object is a tibble.
\end{itemize}

\hypertarget{transform}{%
\chapter{Transform}\label{transform}}

\begin{center}\rule{0.5\linewidth}{\linethickness}\end{center}

``Transformation includes narrowing in on observations of interest (like all people in one city, or all data from the last year), creating new variables that are functions of existing variables (like computing velocity from speed and time), and calculating a set of summary statistics (like counts or means).''\\
- Garrett Grolemund \& Hadley Wickham, \emph{R for Data Science}

\begin{center}\rule{0.5\linewidth}{\linethickness}\end{center}

\hypertarget{arithmetic-summary-statistics}{%
\section{Arithmetic \& Summary Statistics}\label{arithmetic-summary-statistics}}

\texttt{\{assertive\}}

\begin{itemize}
\tightlist
\item
  \texttt{is\_divisible\_by}: Is the input divisible by a number?
\end{itemize}

\texttt{\{base\}}

\begin{itemize}
\tightlist
\item
  \texttt{abs}: Compute absolute value.
\item
  \texttt{colMeans}: Compute the column mean.
\item
  \texttt{colSums}: Compute the column sum.
\item
  \texttt{diff}: Compute the difference between two objects.
\item
  \texttt{IQR}: Compute the inter-quartile range.
\item
  \texttt{max}: Return the maximum value.
\item
  \texttt{mean}: Compute the mean value.

  \begin{itemize}
  \tightlist
  \item
    Note that \texttt{mean} can be used to calculate a percentage when used in \texttt{summarize}, as follows:
  \end{itemize}
\end{itemize}

\begin{Shaded}
\begin{Highlighting}[]
\NormalTok{by_country <-}\StringTok{ }
\StringTok{  }\NormalTok{votes }\OperatorTok\StringTok{ }
\StringTok{  }\KeywordTok{group_by}\NormalTok{(country) }\OperatorTok\StringTok{ }
\StringTok{  }\KeywordTok{summarize}\NormalTok{(}\DataTypeTok{total =} \KeywordTok{n}\NormalTok{(),}
            \DataTypeTok{percent_yes =} \KeywordTok{mean}\NormalTok{(vote }\OperatorTok{==}\StringTok{ }\DecValTok{1}\NormalTok{))}
\end{Highlighting}
\end{Shaded}

\begin{itemize}
\tightlist
\item
  \texttt{median}: Compute the median value.
\item
  \texttt{min}: Return the minimum value.
\item
  Mode: Use \texttt{table} to view the mode of a data set.
\item
  Operators:

  \begin{itemize}
  \tightlist
  \item
    Arithmetic Operators: \texttt{+}, \texttt{-}, \texttt{*}, \texttt{/}, \texttt{\%\%}
  \item
    Comparison Operators: \texttt{\textless{}}, \texttt{\textgreater{}}, \texttt{\textless{}=}, \texttt{\textgreater{}=}, \texttt{==}, \texttt{!=}.

    \begin{itemize}
    \tightlist
    \item
      Use \texttt{identical} and \texttt{all.equal} rather than \texttt{==} and \texttt{!=} in tests where a single \texttt{TRUE} or \texttt{FALSE} is required (such as \texttt{if} expressions).
    \end{itemize}
  \end{itemize}
\item
  \texttt{range}: Return a vector containing the minimum and maximum values.

  \begin{itemize}
  \tightlist
  \item
    Use \texttt{diff(range())} to get the range as a measure of variability.
  \end{itemize}
\item
  \texttt{round}: Round values to a specified number of decimal places.
\item
  \texttt{rowMeans}: Compute the row mean.
\item
  \texttt{rowSums}: Compute the row sum.
\item
  \texttt{sd}: Compute the standard deviation.
\item
  \texttt{signif}: Round values to a specified number of significant digits.
\item
  \texttt{sqrt}: Compute square root.
\item
  \texttt{sum}: Sum elements.
\item
  \texttt{summary}: Compute summary statistics.
\item
  \texttt{var}: Compute the variance.
\end{itemize}

\texttt{\{dplyr\}}

\begin{itemize}
\tightlist
\item
  \texttt{count}: Count/tally observations by group.
\item
  \texttt{group\_by}: Group by one or more variables.
\item
  \texttt{n}: Get the number of observations in a current group.

  \begin{itemize}
  \tightlist
  \item
    Must be used within \texttt{summarise}, \texttt{mutate}, or \texttt{filter}.
  \end{itemize}
\item
  \texttt{n\_distinct}: Count the number of unique values in a vector.
\item
  \texttt{summarize}: Reduce multiple values to a single value.

  \begin{itemize}
  \tightlist
  \item
    Use \texttt{mean(variable\ ==\ value)} to get a percentage (see the above example for \texttt{mean}.)
  \end{itemize}
\item
  \texttt{tally}: An alternative to \texttt{count}.
\item
  \texttt{top\_n}: Select top (or bottom) n rows (by value).
\end{itemize}

\texttt{\{magrittr\}}

\begin{itemize}
\tightlist
\item
  \texttt{extract}: Pipeable extraction operator.

  \begin{itemize}
  \tightlist
  \item
    \texttt{x\ \%\textgreater{}\%\ extract(y)} is equivalent to \texttt{x{[}y{]}}.
  \end{itemize}
\item
  \texttt{multiply\_by}: Pipeable multiplication operator.

  \begin{itemize}
  \tightlist
  \item
    \texttt{x\ \%\textgreater{}\%\ multiply\_by(y)} is equivalent to \texttt{x\ *\ y}.
  \end{itemize}
\item
  \texttt{raise\_to\_power}: Pipeable exponent operator.

  \begin{itemize}
  \tightlist
  \item
    \texttt{x\ \%\textgreater{}\%\ raise\_to\_power(y)} is equivalent to \texttt{x\^{}y}.
  \end{itemize}
\end{itemize}

\texttt{\{stats\}}

\begin{itemize}
\tightlist
\item
  \texttt{aggregate}: Compute summary statistics of data subsets.
\item
  \texttt{cor}: Correlation.
\item
  \texttt{cov}: Covariance.
\item
  \texttt{lag}: Lag a time series.
\item
  \texttt{rnorm}: Generate a random normal distribution.
\item
  \texttt{var}: Variance.
\end{itemize}

\begin{center}\rule{0.5\linewidth}{\linethickness}\end{center}

\hypertarget{create-new-variables-or-modify-existing-ones}{%
\section{Create New Variables or Modify Existing Ones}\label{create-new-variables-or-modify-existing-ones}}

\texttt{countrycode}

\begin{itemize}
\tightlist
\item
  \texttt{countrycode()}: Convert country codes into country names.
\end{itemize}

\texttt{dplyr}

\begin{itemize}
\tightlist
\item
  \texttt{mutate()}: Add new variables.

  \begin{itemize}
  \tightlist
  \item
    \texttt{mutate()} can also be used to modify existing variables. To change the case of a character variable, for example, do something like:
  \end{itemize}
\end{itemize}

\begin{Shaded}
\begin{Highlighting}[]
\NormalTok{df <-}
\StringTok{  }\NormalTok{df }\OperatorTok
\StringTok{  }\KeywordTok{mutate}\NormalTok{(}\DataTypeTok{var_name =} \KeywordTok{str_to_lower}\NormalTok{(var_name))}
\end{Highlighting}
\end{Shaded}

\begin{verbatim}
+ Child function: `transmute()` (drops existing variables).
\end{verbatim}

\begin{itemize}
\tightlist
\item
  \texttt{recode()}: Recode values (the numeric alternative to using \texttt{if\_else} or \texttt{case\_when()}).
\end{itemize}

\begin{center}\rule{0.5\linewidth}{\linethickness}\end{center}

\hypertarget{dates-and-datetimes-1}{%
\section{Dates and Datetimes}\label{dates-and-datetimes-1}}

\texttt{base}

\begin{itemize}
\tightlist
\item
  \texttt{date()}: Get the current system date and time.
\item
  \texttt{difftime()}: Time intervals and differences.

  \begin{itemize}
  \tightlist
  \item
    \texttt{difftime()} is the function behind the \texttt{-} operator when used with dates and datetimes (e.g., \texttt{time\_1\ -\ time\_2} is equivalent to \texttt{difftime(time\_1,\ time\_2)}). The advantage of using \texttt{difftime()} over \texttt{-}, however, is the \texttt{units} argument because it allows you to specify the unit of time in which the difference is calculated.
  \end{itemize}
\item
  \texttt{months()}: Extract the month names.
\item
  \texttt{quarters()}: Extract the calendar quarters.
\item
  \texttt{Sys.Date()}: Get the current date in the current time zone.
\item
  \texttt{Sys.time()}: Get the absolute date-time value (which can be converted to various time zones and may return different days).
\item
  \texttt{weekdays()}: Extract weekday names.
\end{itemize}

\texttt{lubridate}

\begin{itemize}
\tightlist
\item
  \texttt{date()}: Get or set the date component of a date-time.
\item
  \texttt{day()}: Get or set the day component of a datetime.
\item
  \texttt{month()}: Get or set the month component of a datetime.
\item
  \texttt{now()}: The current time (as a POSIXct object).
\item
  \texttt{quarter()}: Get or set the fiscal quarter or semester component of a datetime.
\item
  \texttt{round\_date()}: Round the datetime to the nearest datetime.

  \begin{itemize}
  \tightlist
  \item
    Child functions: \texttt{ceiling\_date()}, \texttt{floor\_date()}.
  \end{itemize}
\item
  Time spans: Duration:

  \begin{itemize}
  \tightlist
  \item
    \texttt{dseconds()}, \texttt{dminutes()}, \texttt{dhours()}, \texttt{ddays()}, \texttt{dweeks()}, \texttt{dyears()}.
  \item
    Use when you are interested in seconds elapsed.
  \end{itemize}
\item
  Time spans: Interval:

  \begin{itemize}
  \tightlist
  \item
    \texttt{interval()}, \texttt{\%-\/-\%}, \texttt{is.interval()}, \texttt{int\_start()}, \texttt{int\_end()}, \texttt{int\_length()}, \texttt{int\_flip()}, \texttt{int\_shift()}, \texttt{int\_overlaps()}, \texttt{int\_standardize()}, \texttt{int\_aligns()}, \texttt{int\_diff()}.
  \item
    Use when you have a start and end.
  \end{itemize}
\item
  Time spans: Period:

  \begin{itemize}
  \tightlist
  \item
    \texttt{seconds()}, \texttt{minutes()}, \texttt{hours()}, \texttt{days()}, \texttt{weeks()}, \texttt{months()}, \texttt{years()}.
  \item
    Use when you are interested in human units.
  \end{itemize}
\item
  Time zones:

  \begin{itemize}
  \tightlist
  \item
    \texttt{force\_tz()}: Change the time zone without changing the clock time.
  \item
    \texttt{tz()}: Extract the time zone from a datetime.
  \item
    \texttt{with\_tz()}: View the same instant in a different time zone.
  \end{itemize}
\item
  \texttt{today()}: The current date (as a Date object).
\item
  \texttt{\%m+\%} \& \texttt{\%m-\%}: Add and subtract months to a date without exceeding the last day of the new month.
\item
  \texttt{\%within\%}: Test whether a date or interval falls within an interval.
\item
  \texttt{year()}: Get or set the year component of a datetime.
\end{itemize}

\begin{center}\rule{0.5\linewidth}{\linethickness}\end{center}

\hypertarget{factors}{%
\section{Factors}\label{factors}}

\texttt{forcats}

\begin{itemize}
\tightlist
\item
  \texttt{fct\_drop()}: Drop levels.
\item
  \texttt{fct\_reorder()}: Reorder levels, based on the value of another variable.
\item
  \texttt{fct\_rev()}: Reverse levels.
\end{itemize}

\texttt{stats}

\begin{itemize}
\tightlist
\item
  \texttt{reorder()}: Reorder levels of a factor.

  \begin{itemize}
  \tightlist
  \item
    Useful within the \texttt{aes()} argument in a \texttt{ggplot()} call.
  \end{itemize}
\end{itemize}

\hypertarget{merge-or-append-data}{%
\section{Merge or Append Data}\label{merge-or-append-data}}

\texttt{base}

\begin{itemize}
\tightlist
\item
  \texttt{append()}: Add elements to a vector.
\item
  \texttt{cbind()}: Combine objects by column.
\item
  \texttt{intersect()}: Combine data shared in common between two datasets.

  \begin{itemize}
  \tightlist
  \item
    Similar to \texttt{dplyr::semi\_join()}.
  \end{itemize}
\item
  \texttt{merge()}: Merge two data frames.

  \begin{itemize}
  \tightlist
  \item
    \texttt{dplyr::join} functions are an alternative to \texttt{merge()}.
  \end{itemize}
\item
  \texttt{rbind()}: Combine objects by row.
\item
  \texttt{setdiff()}: Find the difference between two vectors.

  \begin{itemize}
  \tightlist
  \item
    Similar to \texttt{dplyr::anti\_join()}.
  \end{itemize}
\item
  \texttt{union()}: Combine two datasets without duplicating values.
\end{itemize}

\texttt{dplyr}

\begin{itemize}
\tightlist
\item
  \texttt{bind()}: Bind multiple data frames by row and column.

  \begin{itemize}
  \tightlist
  \item
    Child functions: bind\_rows(), bind\_cols(), combine().
  \end{itemize}
\item
  Join Functions: Join two tables.

  \begin{itemize}
  \tightlist
  \item
    Filtering Joins:

    \begin{itemize}
    \tightlist
    \item
      anti\_join(): Return all rows from \texttt{x} where there are not matching values in \texttt{y}, keeping just columns from \texttt{x}.
    \item
      semi\_join(): Return all rows from \texttt{x} where there are matching values in \texttt{y}, keeping just columns from \texttt{x}. A semi join differs from an inner join because an inner join will return one row of \texttt{x} for each matching row of \texttt{y}, where a semi join will never duplicate rows of \texttt{x}.
    \end{itemize}
  \item
    Mutating Joins:

    \begin{itemize}
    \tightlist
    \item
      full\_join(): Return all rows and all columns from both \texttt{x} and \texttt{y}. Where there are not matching values, returns \texttt{NA} for the one missing.
    \item
      inner\_join(): Return all rows from \texttt{x} where there are matching values in \texttt{y}, and all columns from \texttt{x} and \texttt{y}. If there are multiple matches between \texttt{x} and \texttt{y}, all combination of the matches are returned.
    \item
      left\_join(): Return all rows from \texttt{x}, and all columns from \texttt{x} and \texttt{y}. Rows in \texttt{x} with no match in \texttt{y} will have \texttt{NA} values in the new columns. If there are multiple matches between \texttt{x} and \texttt{y}, all combinations of the matches are returned.
    \item
      right\_join(): Return all rows from \texttt{y}, and all columns from \texttt{x} and \texttt{y}. Rows in \texttt{x} with no match in \texttt{y} will have \texttt{NA} values in the new columns. If there are multiple matches between \texttt{y} and \texttt{x}, all combinations of the matches are returned.
    \end{itemize}
  \end{itemize}
\end{itemize}

\texttt{tibble}

\begin{itemize}
\tightlist
\item
  \texttt{add\_column()}: Add columns to a data frame.
\item
  \texttt{add\_row()}: Add rows to a data frame.
\end{itemize}

\begin{center}\rule{0.5\linewidth}{\linethickness}\end{center}

\hypertarget{narrow-in-on-observations-of-interest}{%
\section{Narrow in on Observations of Interest}\label{narrow-in-on-observations-of-interest}}

\texttt{\{base\}}

\begin{itemize}
\tightlist
\item
  \texttt{droplevels}: Drop unused levels from factors.

  \begin{itemize}
  \tightlist
  \item
    This function will keep levels that have even 1 or 2 counts. If you want to remove levels with low counts from a data set in order to simplify your analysis, first \texttt{filter} out those rows and then use \texttt{droplevels}.
  \end{itemize}
\item
  \texttt{order}: Arrange in ascending or descending order.

  \begin{itemize}
  \tightlist
  \item
    Alternative: \texttt{dplyr::arrange}.
  \end{itemize}
\item
  \texttt{prop.table}: Express table entries as proportions of the marginal table.

  \begin{itemize}
  \tightlist
  \item
    The input is a table produced by \texttt{table}.
  \item
    As these are proportions of the whole, \texttt{sum(prop.table(table\_name))} = 1.
  \item
    Specify conditional proportions on rows or columns by using the \texttt{margin} argument.
  \end{itemize}
\item
  \texttt{table}: Build a table of the counts at each combination of factor levels.

  \begin{itemize}
  \tightlist
  \item
    Use \texttt{prop.table} to see the table entries expressed as proportions.
  \end{itemize}
\end{itemize}

\texttt{\{dplyr\}}

\begin{itemize}
\tightlist
\item
  \texttt{arrange}: Arrange rows by variable, in ascending order.

  \begin{itemize}
  \tightlist
  \item
    Related functions: \texttt{arrange\_all}, \texttt{arrange\_at}, \texttt{arrange\_if}.
  \end{itemize}
\item
  \texttt{distinct}: Select distinct rows.
\item
  \texttt{filter}: Return rows with matching conditions.

  \begin{itemize}
  \tightlist
  \item
    Use \texttt{\%in\%} when using multiple \texttt{\textbar{}} conditions. The following two commands, for example, are equivalent:
  \end{itemize}
\end{itemize}

\begin{Shaded}
\begin{Highlighting}[]
\NormalTok{ilo_data }\OperatorTok\StringTok{ }
\StringTok{  }\KeywordTok{filter}\NormalTok{(country }\OperatorTok\StringTok{ }\KeywordTok{c}\NormalTok{(}\StringTok{"Sweden"}\NormalTok{, }\StringTok{"Switzerland"}\NormalTok{))}

\NormalTok{ilo_data }\OperatorTok\StringTok{ }
\StringTok{  }\KeywordTok{filter}\NormalTok{(country }\OperatorTok{==}\StringTok{ "Sweden"} \OperatorTok{|}\StringTok{ }\NormalTok{country }\OperatorTok{==}\StringTok{ "Switzerland"}\NormalTok{)}
\end{Highlighting}
\end{Shaded}

\begin{itemize}
\tightlist
\item
  \texttt{renam}: Rename variables by name (a modification of \texttt{select}).
\item
  \texttt{sample\_n}: Sample n rows from a table.
\item
  \texttt{select}: Select/rename variables.

  \begin{itemize}
  \tightlist
  \item
    Helper functions include: \texttt{starts\_wit}, \texttt{ends\_with}, \texttt{contains}, \texttt{matches}, \texttt{num\_range}, and \texttt{one\_of}. See \texttt{?dplyr::select\_helpers}.
  \item
    Helper functions take strings (e.g., \texttt{contains("work")} rather than \texttt{contains(work)}).
  \end{itemize}
\item
  \texttt{slice}: Choose rows by position.
\item
  \texttt{transmute}: Create or transform variables.

  \begin{itemize}
  \tightlist
  \item
    Like a combination of \texttt{select} and \texttt{mutate}.
  \end{itemize}
\end{itemize}

\begin{center}\rule{0.5\linewidth}{\linethickness}\end{center}

\hypertarget{test}{%
\section{Test}\label{test}}

\texttt{base}

\begin{itemize}
\tightlist
\item
  \texttt{identical()}: Test objects for exact equality.
\item
  \texttt{match()}: Value matching.

  \begin{itemize}
  \tightlist
  \item
    \texttt{\%in\%} is the more intuitive binary operator.
  \end{itemize}
\item
  \texttt{setequal()}: Check two vectors for equality.
\item
  \texttt{which()}: Determine which indices are \texttt{TRUE}.

  \begin{itemize}
  \tightlist
  \item
    This function is often unnecessary, according to the DataCamp article \href{https://www.datacamp.com/community/tutorials/five-tips-r-code-improve}{``Five Tips to Improve Your R Code''}.
  \end{itemize}
\item
  \texttt{which.min()}/\texttt{which.max()}: Where is the \texttt{min()}/\texttt{max()} or first \texttt{TRUE}/\texttt{FALSE}?
\item
  \texttt{\%in\%}: See \texttt{match()}.
\end{itemize}

\hypertarget{visualize}{%
\chapter{Visualize}\label{visualize}}

\begin{center}\rule{0.5\linewidth}{\linethickness}\end{center}

``Visualisation is a fundamentally human activity. A good visualisation will show you things that you did not expect, or raise new questions about the data. A good visualisation might also hint that you're asking the wrong question, or you need to collect different data. Visualisations can surprise you, but don't scale particularly well because they require a human to interpret them.''\\
- Garrett Grolemund \& Hadley Wickham, \emph{R for Data Science}

\begin{center}\rule{0.5\linewidth}{\linethickness}\end{center}

\hypertarget{flowcharts}{%
\section{Flowcharts}\label{flowcharts}}

\texttt{diagram}

\texttt{DiagrammeR}

\begin{center}\rule{0.5\linewidth}{\linethickness}\end{center}

\hypertarget{interfaces}{%
\section{Interfaces}\label{interfaces}}

\texttt{shiny}

\begin{center}\rule{0.5\linewidth}{\linethickness}\end{center}

\hypertarget{plots}{%
\section{Plots}\label{plots}}

\texttt{\{base\}}

\begin{itemize}
\tightlist
\item
  \texttt{abline}: Add straight lines to a plot.
\item
  \texttt{plot}: Generic X-Y plotting.
\item
  \texttt{points}: Add points to a plot.
\end{itemize}

\texttt{\{ggbeeswarm\}}

\begin{itemize}
\tightlist
\item
  \texttt{geom\_beeswarm}: Create a beeswarm plot.
\end{itemize}

\texttt{ggplot2}

\begin{itemize}
\tightlist
\item
  \texttt{aes}: Construct aesthetic mappings.

  \begin{itemize}
  \tightlist
  \item
    Arguments include \texttt{color}, \texttt{fill}, \texttt{size}, \texttt{labels}, \texttt{alpha}, \texttt{shape} (1-20 accept color attributes and 21-25 accept color and fill attributes), \texttt{linewidth}, \texttt{linetype}, and \texttt{group}.
  \item
    Use \texttt{?pch} to see options for \texttt{shape}.
  \item
    The default shape for points does not have a \texttt{fill} attribute, which means that mapping a categorical variable onto \texttt{fill} won't result in multiple colors.
  \item
    Helper functions to include in the call when needing to modify the data include: \texttt{stats::reorder}.
  \end{itemize}
\item
  \texttt{coord\_*}

  \begin{itemize}
  \tightlist
  \item
    \texttt{coord\_cartesian}: Zoom a plot in or out without changing the underlying data.
  \item
    \texttt{coord\_flip}: Flip the x and y axes.
  \item
    \texttt{coord\_polar}: Used to convert a stacked bar chart to a pie chart.
  \end{itemize}
\item
  \texttt{facet\_*}

  \begin{itemize}
  \tightlist
  \item
    \texttt{facet\_grid}: Lay out panels in a grid.
  \item
    \texttt{facet\_wrap}: Wrap a 1D ribbon of panels into 2D (observe a variable, conditional on another variable).
  \end{itemize}
\item
  \texttt{geom\_*}: Create a geometry.

  \begin{itemize}
  \tightlist
  \item
    Note that \texttt{aes} can be called within \texttt{geom\_*} rather than prior to.
  \item
    \texttt{geom\_abline}: Add reference lines to a plot.
  \item
    \texttt{geom\_bar}: Create a bar chart, where the height of the bar is proportional to the number of cases in each group.
  \item
    \texttt{geom\_boxplot}: Create a boxplot.
  \item
    \texttt{geom\_col}: Create a bar chart, where the height of the bar represents values in the data.
  \item
    \texttt{geom\_density}: Create a kernal density estimate (a smoothed version of a histogram).

    \begin{itemize}
    \tightlist
    \item
      Consider using \texttt{geom\_rug} with \texttt{geom\_density} in order to be transparent about smoothed data.
    \end{itemize}
  \item
    \texttt{geom\_dotplot}: Create a histogram out of dots.
  \item
    \texttt{geom\_histogram}: Create a histogram.

    \begin{itemize}
    \tightlist
    \item
      Use \texttt{y\ =\ stat(density)} to rescale the y-axis from counts to a probability estimate.
    \end{itemize}
  \item
    \texttt{geom\_hline}: Add a horizontal reference line to the plot.
    *\texttt{geom\_jitter}: Jitter points.

    \begin{itemize}
    \tightlist
    \item
      Useful with \texttt{geom\_boxplot}.
    \end{itemize}
  \item
    \texttt{geom\_path}: Connect observations in the order in which they appear.
  \item
    \texttt{geom\_point}: Create a scatterplot (a.k.a. point chart or dot plot).
  \item
    \texttt{geom\_rug}: Create a ruge plot in the margin.

    \begin{itemize}
    \tightlist
    \item
      Useful with \texttt{geom\_density}.
    \end{itemize}
  \item
    \texttt{geom\_smooth}: Smoothed conditional means; aids the eye in seeing patterns in the presence of overplotting.
  \item
    \texttt{geom\_text}: Add text directly to the plot.
  \item
    \texttt{geom\_violin}: Create a violin plot.
  \item
    \texttt{geom\_vline}: Add a vertical reference line to the plot.
  \end{itemize}
\item
  \texttt{ggplot}: Create a plot.
\item
  \texttt{ggtitle}: Modify the plot title.
\item
  \texttt{group}: Specify groupings.

  \begin{itemize}
  \tightlist
  \item
    Used within \texttt{aes}. Note that \texttt{group} is usually unnecessary when specifying \texttt{color}, \texttt{shape}, \texttt{fill}, or \texttt{linetype} within \texttt{aes}, or when using facets. See \href{https://ggplot2.tidyverse.org/reference/aes_group_order.html}{``Aesthetics: grouping''} for more information.
  \end{itemize}
\item
  \texttt{labs}: Modify axis, legend, and plot labels.
\item
  \texttt{position\_*}:

  \begin{itemize}
  \tightlist
  \item
    \texttt{position\_identity}: Don't adjust position.
  \item
    \texttt{position\_dodge}: Dodge overlapping objects side-to-side.
  \item
    \texttt{position\_nudge}: Nudge points a fixed distance.
  \item
    \texttt{position\_stack}: Stack overlapping objects on top of each other, as counts.
  \item
    \texttt{position\_fill}: Stack overlapping objects on top of each other, as densities.
  \item
    \texttt{position\_jitter}: Jitter points to avoid overplotting.
  \item
    \texttt{position\_jitterdodge}: Simultaneously dodge and jitter.
  \end{itemize}
\item
  \texttt{scale\_*\_*}:

  \begin{itemize}
  \tightlist
  \item
    \texttt{scale\_x\_*}
  \item
    \texttt{scale\_y\_*}
  \item
    \texttt{scale\_color\_*}
  \item
    \texttt{scale\_fill\_*}
  \item
    \texttt{scale\_shape\_*}
  \item
    \texttt{scale\_size}
  \item
    \texttt{scale\_linetype\_*}
  \end{itemize}
\item
  \texttt{stat\_*}: Statistic layers (sometimes called by \texttt{geom\_} layers).

  \begin{itemize}
  \tightlist
  \item
    \texttt{stat\_bin}
  \item
    \texttt{stat\_bin2d}
  \item
    \texttt{stat\_bindot}
  \item
    \texttt{stat\_binhex}
  \item
    \texttt{stat\_boxplot}
  \item
    \texttt{stat\_contour}
  \item
    \texttt{stat\_quantile}
  \item
    \texttt{stat\_smooth}
  \item
    \texttt{stat\_sum}
  \end{itemize}
\item
  \texttt{xlab}: Modify the label of the x-axis.
\item
  \texttt{ylab}: Modify the label of the y-axis.
\end{itemize}

\texttt{\{ggridges\}}

\begin{itemize}
\tightlist
\item
  \texttt{geom\_density\_ridges}: Create a ridgeline plot.
\end{itemize}

\texttt{\{graphics\}}

\begin{itemize}
\tightlist
\item
  \texttt{boxplot}: Create a box-and-whisker plot.
\item
  \texttt{hist}: Create a histogram.
\item
  \texttt{stripchart}: One dimensional scatter plots.

  \begin{itemize}
  \tightlist
  \item
    Preferable to \texttt{ggplot} when creating one-dimensional plots.
  \end{itemize}
\end{itemize}

\texttt{\{grDevices\}}

\begin{itemize}
\tightlist
\item
  \texttt{colorRamp}: Color interpolation.
\item
  \texttt{colorRampPalette}: Color interpolation.
\end{itemize}

\texttt{\{RColorBrewer\}}

\begin{itemize}
\tightlist
\item
  \texttt{brewer.pal}: Make the ColorBrewer color palettes available as R palattes.
\end{itemize}

\hypertarget{model}{%
\chapter{Model}\label{model}}

\begin{center}\rule{0.5\linewidth}{\linethickness}\end{center}

``Models are complementary tools to visualisation. Once you have made your questions sufficiently precise, you can use a model to answer them. Models are a fundamentally mathematical or computational tool, so they generally scale well. \ldots{} But every model makes assumptions, and by its very nature a model cannot question its own assumptions. That means a model cannot fundamentally surprise you.
- Garrett Grolemund \& Hadley Wickham, \emph{R for Data Science}''

\begin{center}\rule{0.5\linewidth}{\linethickness}\end{center}

\hypertarget{general}{%
\section{General}\label{general}}

\texttt{\{base\}}

\begin{itemize}
\tightlist
\item
  \texttt{sample}: Random samples and permutations.
\end{itemize}

\texttt{\{dplyr\}}

\begin{itemize}
\tightlist
\item
  \texttt{sample\_n}: Sample n rows from a table.
\end{itemize}

\texttt{\{stats\}}

\begin{itemize}
\tightlist
\item
  \texttt{coef}: Extract model coefficients.
\item
  \texttt{cor}: Correlation.
\item
  \texttt{cov}: Covariance.
\item
  \texttt{cov2cor}: Scale a covariance matrix into a correlation matrix.
\item
  \texttt{var}: Variance.
\end{itemize}

\begin{center}\rule{0.5\linewidth}{\linethickness}\end{center}

\hypertarget{regression}{%
\section{Regression}\label{regression}}

\texttt{\{broom\}}

\begin{itemize}
\tightlist
\item
  \texttt{augment}: Augment data with information from an object.
\item
  \texttt{glance}: Construct a single row summary of a model, fit, or other object.
\item
  \texttt{tidy}: Turn an object into a tidy tibble.
\end{itemize}

\texttt{\{mgcv\}}

\begin{itemize}
\tightlist
\item
  \texttt{gam}: Generalized additive models (GAMs) with integrated smoothness estimation.
\end{itemize}

\texttt{\{stats\}}

\begin{itemize}
\tightlist
\item
  \texttt{df.residual}: Get the residual degrees of freedom.
\item
  \texttt{lm}: Fit linear models.
\item
  \texttt{p.adjust}: Adjust p-values for multiple comparisons.
\item
  \texttt{predict}: Model predictions.
\item
  \texttt{residuals}: Extract model residuals.
\end{itemize}

\texttt{\{tidyr\}}

\begin{itemize}
\tightlist
\item
  \texttt{nest}: Nest repeated values in a list-variable.

  \begin{itemize}
  \tightlist
  \item
    Helpful when separating a data frame in preparation to model the data for each grouping.
  \end{itemize}
\end{itemize}

\begin{center}\rule{0.5\linewidth}{\linethickness}\end{center}

\hypertarget{simulation-prediction}{%
\section{Simulation \& Prediction}\label{simulation-prediction}}

\texttt{base}

\begin{itemize}
\tightlist
\item
  \texttt{set.seed()}: Random number generation.
\end{itemize}

\texttt{stats}

\begin{itemize}
\tightlist
\item
  \texttt{predict()}: Model prediction.

  \begin{itemize}
  \tightlist
  \item
    Use with ``lm'' class objects and new data to predict new values (e.g., \texttt{predict(model,\ newdata)}).
  \end{itemize}
\item
  \texttt{rnorm()}: Generate a normal distribution.
\end{itemize}

\begin{center}\rule{0.5\linewidth}{\linethickness}\end{center}

\hypertarget{strings-1}{%
\section{Strings}\label{strings-1}}

\texttt{base}

\begin{itemize}
\tightlist
\item
  \texttt{agrep()}: Approximate string matching (fuzzy matching).
\end{itemize}

\texttt{fuzzyjoin}

\begin{itemize}
\tightlist
\item
  \texttt{stringdist\_join()}: Join two tables based on fuzzy string matching of their columns.

  \begin{itemize}
  \tightlist
  \item
    Child functions: \texttt{stringdist\_inner\_join()}, \texttt{stringdist\_left\_join()}, \texttt{stringdist\_right\_join()}, \texttt{stringdist\_full\_join()}, \texttt{stringdist\_semi\_join()}, \texttt{stringdist\_anti\_join()}.
  \end{itemize}
\end{itemize}

\texttt{fuzzywuzzyR}

\begin{itemize}
\tightlist
\item
  \texttt{FuzzMatcher()}: Fuzzy character string matching (ratios).
\end{itemize}

\hypertarget{communicate}{%
\chapter{Communicate}\label{communicate}}

\begin{center}\rule{0.5\linewidth}{\linethickness}\end{center}

``The last step of data science is communication, an absolutely critical part of any data analysis project. It doesn't matter how well your models and visualisation have led you to understand the data unless you can also communicate your results to others.''\\
- Garrett Grolemund \& Hadley Wickham, \emph{R for Data Science}

\begin{center}\rule{0.5\linewidth}{\linethickness}\end{center}

\hypertarget{export}{%
\section{Export}\label{export}}

\texttt{base}

\begin{itemize}
\tightlist
\item
  \texttt{file.path()}: Construct a file path.
\item
  \texttt{print()}: Print values.

  \begin{itemize}
  \tightlist
  \item
    Use the \texttt{include.rownames\ =\ FALSE} argument to remove row numbers (or names) from the output.
  \end{itemize}
\item
  \texttt{save()}: Save R objects.
\item
  \texttt{saveRDS()}: Save a single R object.

  \begin{itemize}
  \tightlist
  \item
    See \href{https://www.fromthebottomoftheheap.net/2012/04/01/saving-and-loading-r-objects/}{``A better way of saving and loading objects in R''} to understand the differences between \texttt{save()} and \texttt{saveRDS()}.
  \end{itemize}
\end{itemize}

\texttt{readr}

\begin{itemize}
\tightlist
\item
  \texttt{write\_delim()}: Write a data frame to a delimited file.

  \begin{itemize}
  \tightlist
  \item
    About twice as fast as \texttt{write.csv()} and never writes row names.
  \item
    Child functions: \texttt{write\_csv()}, \texttt{write\_excel\_csv()}, \texttt{write\_tsv()}.
  \end{itemize}
\end{itemize}

\texttt{utils}

\begin{itemize}
\tightlist
\item
  \texttt{write.table()}: Data output.

  \begin{itemize}
  \tightlist
  \item
    Prefer \texttt{readr::write\_delim()} to \texttt{utils::write.table()}.
  \item
    Child functions: \texttt{write.csv()}, \texttt{write.csv2()}.
  \end{itemize}
\end{itemize}

\texttt{XLConnect}: Read, write, and format Excel data.
\#\# Format Output

\texttt{base}

\begin{itemize}
\tightlist
\item
  \texttt{format()}: Format an object for pretty printing.
\end{itemize}

\texttt{knitr}

\begin{itemize}
\tightlist
\item
  \texttt{kable()}: Create tables in LaTex, HTML, Markdown, and reStructuredText.
\end{itemize}

\texttt{lubridate}

\begin{itemize}
\tightlist
\item
  \texttt{stamp()}: Format dates and times based on human-friendly templates.
\end{itemize}

\texttt{scales}: Scale functions for visualization.

\begin{itemize}
\tightlist
\item
  \texttt{dollar()}: Round to the nearest cent and display dollar sign.
\end{itemize}

\begin{center}\rule{0.5\linewidth}{\linethickness}\end{center}

\hypertarget{graphics}{%
\section{Graphics}\label{graphics}}

`knitr'

\begin{itemize}
\tightlist
\item
  \texttt{include\_graphics()}: Embed external images in `knitr' documents.

  \begin{itemize}
  \tightlist
  \item
    Preferable to the \texttt{!{[}alt\ text\ or\ image\ title{]}(path/to/image)} Markdown syntax for embedding an external image, as \texttt{include\_graphics()} offers more control over the attributes of the image.
  \end{itemize}
\end{itemize}

\begin{center}\rule{0.5\linewidth}{\linethickness}\end{center}

\hypertarget{plots-1}{%
\section{Plots}\label{plots-1}}

\texttt{ggplot2}

\begin{itemize}
\tightlist
\item
  \texttt{element\_*()}: Specify the display of how non-data components of a plot are drawn.
\item
  \texttt{labs()}: Modify axis, legend, and plot labels.

  \begin{itemize}
  \tightlist
  \item
    Child functions: \texttt{xlab()}, \texttt{ylab()}, \texttt{ggtitle()}
  \end{itemize}
\item
  \texttt{theme()}: Customize the non-data components of a plot.
\end{itemize}

\begin{Shaded}
\begin{Highlighting}[]
\KeywordTok{ggplot}\NormalTok{(plot_data_}\DecValTok{2006}\NormalTok{) }\OperatorTok{+}
\StringTok{  }\KeywordTok{geom_histogram}\NormalTok{(}\KeywordTok{aes}\NormalTok{(}\DataTypeTok{x =}\NormalTok{ working_hours)) }\OperatorTok{+}
\StringTok{  }\KeywordTok{labs}\NormalTok{(}\DataTypeTok{x =} \StringTok{"Working hours per week"}\NormalTok{,}
       \DataTypeTok{y =} \StringTok{"Number of countries"}\NormalTok{) }\OperatorTok{+}
\StringTok{  }\KeywordTok{theme}\NormalTok{(}
    \DataTypeTok{text =} \KeywordTok{element_text}\NormalTok{(}\DataTypeTok{family =} \StringTok{"Bookman"}\NormalTok{, }\DataTypeTok{color =} \StringTok{"gray25"}\NormalTok{))}
\end{Highlighting}
\end{Shaded}

\begin{verbatim}
+ child Functions: `theme_*()` are predefined themes, including `theme_classic()` and `theme_minimal()`. See `?theme_classic` for a list of predefined themes.
\end{verbatim}

\begin{center}\rule{0.5\linewidth}{\linethickness}\end{center}

\hypertarget{rmarkdown}{%
\section{RMarkdown}\label{rmarkdown}}

\begin{itemize}
\tightlist
\item
  See \href{https://bookdown.org/yihui/rmarkdown/}{RMarkdown: The Definitive Guide}
\item
  See \href{https://developer.mozilla.org/en-US/docs/Web/CSS/CSS_Selectors}{Mozzila Developer Network} for CSS help.
\end{itemize}

\hypertarget{program}{%
\chapter{Program}\label{program}}

\begin{center}\rule{0.5\linewidth}{\linethickness}\end{center}

``Surrounding {[}the tools for importing, tidying, transforming, visualising, modeling, and communicating data{]} is programming. Programming is a cross-cutting tool that you use in every part of a project. You don't need to be an expert programmer to be a data scientist, but learning more about programming pays off because becoming a better programmer allows you to automate common tasks, and solve new problems with greater ease.''\\
- Garrett Grolemund \& Hadley Wickham, \emph{R for Data Science}

\begin{center}\rule{0.5\linewidth}{\linethickness}\end{center}

\hypertarget{conditionals-control-flows}{%
\section{Conditionals \& Control Flows}\label{conditionals-control-flows}}

\texttt{base}

\begin{itemize}
\tightlist
\item
  Control (access documentation using ``?Control'')

  \begin{itemize}
  \tightlist
  \item
    \texttt{if\ (cond)\ expr}

    \begin{itemize}
    \tightlist
    \item
      The key difference between \texttt{if\ (cond)\ expr} and \texttt{ifelse} is that \texttt{if\ (cond)\ expr} will evaluate only the first element of an object with \texttt{length\ \textgreater{}\ 1}. See the documentation for each function and the ateucher's GitHub example titled \href{https://gist.github.com/ateucher/c7359f566eded9fcd4a255f4cbd4fe67}{``R: if vs ifelse''} to learn more.
    \end{itemize}
  \item
    \texttt{for\ (var\ in\ seq)\ expr}
  \item
    \texttt{while\ (cond)\ expr}
  \item
    \texttt{repeat\ expr}
  \item
    \texttt{break}
  \item
    \texttt{next}
  \end{itemize}
\item
  \texttt{identical()}: Test objects for exact equality.

  \begin{itemize}
  \tightlist
  \item
    Use \texttt{identical()} rather than \texttt{==} and \texttt{!=} in \texttt{if} and \texttt{while} statements to test for equality.
  \end{itemize}
\item
  'ifelse()\texttt{:\ Conditional\ element\ selection.\ \ \ +}dplyr::if\_else()` is more strict by checking the object type.
\item
  \texttt{stop()}: Stop execution of the expression and execute an error action.

  \begin{itemize}
  \tightlist
  \item
    Useful in combination with an \texttt{if} statement when you want to generate helpful error messages.
  \end{itemize}
\item
  \texttt{stopifnot()}: Ensure the truth of an R expression.

  \begin{itemize}
  \tightlist
  \item
    Prefer \texttt{base::stop()} to \texttt{base::stopifnot()}.
  \end{itemize}
\end{itemize}

\texttt{dplyr}

\begin{itemize}
\tightlist
\item
  \texttt{case\_when()}: A general vectorized \texttt{if}.
\item
  \texttt{if\_else()}: Vectorized \texttt{if}.
\end{itemize}

\begin{center}\rule{0.5\linewidth}{\linethickness}\end{center}

\hypertarget{environment-and-workspace}{%
\section{Environment and Workspace}\label{environment-and-workspace}}

\texttt{\{base\}}

\begin{itemize}
\tightlist
\item
  \texttt{dir}: List the files in a directory/folder.
\item
  Environments

  \begin{itemize}
  \tightlist
  \item
    \texttt{baseenv}: The environment of the \texttt{base} package, it's enclosing environment (``parent environment'') is the empty environment.
  \item
    \texttt{emptyenv}: The empty environment, which is the ancestor of all environments and the only environment without an enclosing environment.
  \item
    \texttt{environment}: The current environment.
  \item
    \texttt{globalenv}: The environment in which you normally work, it's enclosing environment is the last package attached with \texttt{library} or \texttt{require}.
  \item
    \texttt{new.env}: Create a new environment.
  \end{itemize}
\item
  \texttt{environmentName}: Return the name of the environment, as a character string.
\item
  \texttt{exists}: Look for an R object of the given name and possibly return it.

  \begin{itemize}
  \tightlist
  \item
    Must use quotations to name the object.
  \item
    Remember that \texttt{R} will look for an object in parent environments until it reaches the empty environment, so use \texttt{inherits\ =\ FALSE} to limit the search to only the current environment.
  \end{itemize}
\item
  \texttt{getOption}: Set and examine global options.
\item
  \texttt{getwd}: Get the working directory.
\item
  \texttt{history}: Display the previous 25 commands.
\item
  \texttt{install.packages}: Install packages from repositories or local files.
\item
  \texttt{library}: Load and attach packages, returning an error if the packages does not exist.
\item
  \texttt{list.files}: List the files in a directory/folder.
\item
  `loadedNamespaces: Return the loaded name spaces.
\item
  \texttt{loadhistory}: Recall command history.
\item
  \texttt{ls}: List objects in the specified environment.
\item
  \texttt{list2env}: From a list, build or add to an environment.
\item
  \texttt{options}: Set and examine global options.
\item
  \texttt{parent.env}: Return the enclosing environment of the environment listed as an argument.

  \begin{itemize}
  \tightlist
  \item
    \texttt{parent.env} returns information that can be unhelpful, so use with \texttt{environmentName}, as follows: \texttt{parent.env(environment\_name)\ \%\textgreater{}\%\ environmentName}.
  \end{itemize}
\item
  \texttt{q}: Terminate an R session.
\item
  \texttt{R.version}: Version information.
\item
  \texttt{R.version.string}: Version information.

  \begin{itemize}
  \tightlist
  \item
    Same call as \texttt{R.version\$version.string}.
  \end{itemize}
\item
  \texttt{require}: Load and attach packages, returning \texttt{FALSE} if the package does not exist.
\item
  \texttt{rm}: Remove objects from a specified environment.
\item
  \texttt{savehistory}: Save command history (default value is ``.Rhistory'').
\item
  \texttt{save.image}: Save the current workspace.
\item
  \texttt{search}: Return a list of attached packages and R objects.
\item
  \texttt{searchpaths}: Return the path to attached packages.
\item
  \texttt{setwd()}: Set the working directory file path.

  \begin{itemize}
  \tightlist
  \item
    When using Windows, use ``/'' instead of "".
  \end{itemize}
\item
  \texttt{Sys.info}: Extract system and user information.

  \begin{itemize}
  \tightlist
  \item
    Example: \texttt{Sys.info(){[}c("sysname",\ "release"){]}}.
  \end{itemize}
\end{itemize}

\texttt{\{gdata\}}

\begin{itemize}
\tightlist
\item
  \texttt{object.size}: Report the space allocated for an object.

  \begin{itemize}
  \tightlist
  \item
    See also \texttt{utils::object.size}.
  \end{itemize}
\end{itemize}

\texttt{\{installr\}}

\begin{itemize}
\tightlist
\item
  \texttt{updateR}: Check for the latest R version; downloads and installs new R versions.
\end{itemize}

\texttt{\{pryr\}}

\begin{itemize}
\tightlist
\item
  \texttt{where}: Find where a name is defined.
\end{itemize}

\texttt{\{utils\}}

\begin{itemize}
\tightlist
\item
  \texttt{ls.str}: List objects and their structure.
\item
  \texttt{object.size}: Report the space allocated for an object.

  \begin{itemize}
  \tightlist
  \item
    See also \texttt{gdata::object.size}.
  \end{itemize}
\item
  \texttt{sessionInfo}: Collect information about the current R session.
\end{itemize}

References:

\begin{itemize}
\tightlist
\item
  \href{http://adv-r.had.co.nz/Environments.html\#environments}{``Environments''} (Hadley Wickham, \href{http://adv-r.had.co.nz/}{\emph{Advanced R}})
\end{itemize}

\begin{center}\rule{0.5\linewidth}{\linethickness}\end{center}

\hypertarget{evaluation-standard-and-non-standard}{%
\section{Evaluation (Standard and Non-standard)}\label{evaluation-standard-and-non-standard}}

\texttt{base}

\begin{itemize}
\tightlist
\item
  \texttt{cat()}: Concatenate and print.
\item
  \texttt{print():\ Print\ the\ argument\ to\ the\ Console.\ \ \ +\ A\ shortcut\ to}print()` is to place the code you want printed inside parentheses.
\item
  \texttt{quote()}: Return the argument, unevaluated.
\item
  \texttt{writeLines()}: Display quotes and backslashes as they would be read, rather than as R stores them (i.e., see the raw contents of the string, as the \texttt{print()} representation is not the same as the string itself).
\end{itemize}

\texttt{rlang}

\begin{itemize}
\tightlist
\item
  Quosures

  \begin{itemize}
  \tightlist
  \item
    \texttt{enquo()}, \texttt{new\_quosure()}, \texttt{quo()}.
  \end{itemize}
\end{itemize}

References:

\begin{itemize}
\tightlist
\item
  \href{http://adv-r.had.co.nz/Computing-on-the-language.html}{``Non-standard evaluation''} (Hadley Wickham, \href{http://adv-r.had.co.nz/}{\emph{Advanced R}})
\item
  \href{https://cran.r-project.org/web/packages/lazyeval/vignettes/lazyeval.html}{``Non-standard evaluation''} (Hadley Wickham, \texttt{lazyeval} package vignette)
\item
  \href{https://dplyr.tidyverse.org/articles/programming.html}{``Programming with dplyr''} (dplyr.tidyverse.org)
\end{itemize}

\begin{center}\rule{0.5\linewidth}{\linethickness}\end{center}

\hypertarget{functionals}{%
\section{Functionals}\label{functionals}}

\texttt{\{base\}}

\begin{itemize}
\tightlist
\item
  Apply Functions

  \begin{itemize}
  \tightlist
  \item
    \texttt{apply}: Apply functions over array margins.
  \item
    \texttt{lapply}: Apply a function over a list or vector.
  \item
    \texttt{sapply}: Apply a function over a list or vector and return a vector or matrix.
  \item
    \texttt{vapply}: A safer version of \texttt{sapply}, as it requires the output type to be predetermined.
  \item
    \texttt{mapply}: Apply a function to multiple list or vector arguments.
  \item
    \texttt{rapply}: Recursively apply a function to a list.
  \item
    \texttt{tapply}: Apply a function over a ragged array.
  \end{itemize}
\end{itemize}

\texttt{\{purrr\}}

\begin{itemize}
\tightlist
\item
  \texttt{map}: Apply a function to each element of a vector.

  \begin{itemize}
  \tightlist
  \item
    \texttt{tidyr::unnest} is useful in changing the list-column output of \texttt{map} into rows.
  \end{itemize}
\item
  \texttt{map2}: Map over multiple inputs simultaneously.
\item
  \texttt{map\_if}: Apply a function to elements of that match a condition.
\item
  \texttt{possibly}: Usese a default value whenever an error occurs.
\item
  \texttt{quietly}: Capture side effects in a list with components \texttt{result}, \texttt{output}, \texttt{messages}, and \texttt{warnings}.
\item
  \texttt{safely}: Capture side effects in a list with components \texttt{result} and \texttt{error}.
\item
  \texttt{transpose}: Transpose a list (turn a list-of-lists inside-out).
\end{itemize}

\begin{center}\rule{0.5\linewidth}{\linethickness}\end{center}

\hypertarget{functions}{%
\section{Functions}\label{functions}}

\texttt{\{assertive\}}

\begin{itemize}
\tightlist
\item
  \texttt{assert\_*}: Check whether the input is \texttt{*} (e..g, \texttt{assert\_is\_numeric}) and throw and error if the input does not meet the condition.
\item
  \texttt{coerce\_to}: Coerce the input to a different class, with a warning.
\item
  \texttt{is\_*}: Checks whether the input matches the condition specified by \texttt{*} (e.g., \texttt{is\_non\_positive}).
\item
  \texttt{use\_first}: Use only the first element of a vector.
\end{itemize}

\texttt{\{base\}}

\begin{itemize}
\tightlist
\item
  \texttt{do.call}: Execute a function call from a name or a function and a list of arguments to be passed to the function.
\item
  \texttt{invisible}: Return a (temporarily) invisible copy of an object.
\item
  \texttt{match.arg}: Argument verification.

  \begin{itemize}
  \tightlist
  \item
    Useful when matching a character argument specified in the function signature. For example,
  \end{itemize}
\end{itemize}

\begin{Shaded}
\begin{Highlighting}[]
\KeywordTok{args}\NormalTok{(prop.test)}
\end{Highlighting}
\end{Shaded}

\begin{verbatim}
## function (x, n, p = NULL, alternative = c("two.sided", "less", 
##     "greater"), conf.level = 0.95, correct = TRUE) 
## NULL
\end{verbatim}

\begin{Shaded}
\begin{Highlighting}[]
\CommentTok{# The body of `prop.test` contains the following line of code:}
\CommentTok{# `alternative <- match.arg(alternative), which reassigns it to the selected}
\CommentTok{# character vector.}
\end{Highlighting}
\end{Shaded}

\begin{itemize}
\tightlist
\item
  \texttt{message}: Generage a diagnostic message.

  \begin{itemize}
  \tightlist
  \item
    Preferable to generating a message using \texttt{cat}.
  \end{itemize}
\item
  \texttt{return}: Return a value from a function.

  \begin{itemize}
  \tightlist
  \item
    Useful in \texttt{if} statements where one condition is simple and the other is complex (see section 19.6.1 ``Explicit return statements'' in Hadley Wickham's \href{https://r4ds.had.co.nz/functions.html}{\emph{R for Data Science}}.
  \end{itemize}
\item
  \texttt{setNames}: Set the names in an object.

  \begin{itemize}
  \tightlist
  \item
    Useful in function writing; see documentation.
  \end{itemize}
\item
  \texttt{stop}: Stop execution of the expression and execute an error action.

  \begin{itemize}
  \tightlist
  \item
    Useful in combination with an \texttt{if} statement when you want to generate helpful error messages.
  \end{itemize}
\item
  \texttt{stopifnot}: Ensure the truth of an R expression.

  \begin{itemize}
  \tightlist
  \item
    See section 19.5.2 ``Checking values'' in Hadley Wickham's \href{https://r4ds.had.co.nz/functions.html}{\emph{R for Data Science}} for a discussion of \texttt{stop} versus \texttt{stopifnot}.
  \item
    Consider functions from \texttt{\{assertive\}} as an alternative to \texttt{stopifnot} and \texttt{stop}.
  \end{itemize}
\item
  \texttt{unlist}: Flatten lists.

  \begin{itemize}
  \tightlist
  \item
    Useful when using \texttt{purrr}'s \texttt{map} functions, which return objects as type \texttt{list}.
  \end{itemize}
\end{itemize}

\texttt{\{zeallot\}}

\begin{itemize}
\tightlist
\item
  \texttt{\%\textless{}-\%}: Multiple assignment operator.

  \begin{itemize}
  \tightlist
  \item
    Useful when you want to return multiple output from a function. For example:
  \end{itemize}
\end{itemize}

\begin{Shaded}
\begin{Highlighting}[]
\NormalTok{session <-}\StringTok{ }\ControlFlowTok{function}\NormalTok{() \{}
  
  \KeywordTok{list}\NormalTok{(}
    \DataTypeTok{r_version =}\NormalTok{ R.version.string,}
    \DataTypeTok{operating_system =} \KeywordTok{Sys.info}\NormalTok{()[}\KeywordTok{c}\NormalTok{(}\StringTok{"sysname"}\NormalTok{, }\StringTok{"release"}\NormalTok{)],}
    \DataTypeTok{loaded_pkgs =} \KeywordTok{loadedNamespaces}\NormalTok{()}
\NormalTok{  )}
\NormalTok{\}}

\KeywordTok{c}\NormalTok{(vrsn, os, pkgs) }\OperatorTok\StringTok{ }\KeywordTok{session}\NormalTok{() }
\end{Highlighting}
\end{Shaded}

\begin{center}\rule{0.5\linewidth}{\linethickness}\end{center}

\hypertarget{learn-about-an-object}{%
\section{Learn About an Object}\label{learn-about-an-object}}

\texttt{base}

\begin{itemize}
\tightlist
\item
  \texttt{args()}: Display the argument names and default values of a function.
\item
  \texttt{attributes()}: View or assign an objects attributes (e.g., \texttt{class()}, \texttt{dim()}, \texttt{dimnames()}, \texttt{names()}, \texttt{row.names()}).
\item
  \texttt{body()}: Get or set the body of a function.
\item
  \texttt{colnames()}: Retrieve or set column names.
\item
  \texttt{dim()}: Retrieve or set the dimnames of an object.
\item
  \texttt{dimnames()}: Retrieve or set the dimension names of an object.
\item
  \texttt{formals()}: Get or set the formal arguments of a function.
\item
  \texttt{help()}: Get the topic documentation.
\item
  \texttt{help.search()}: Search the help system for documentation matching a given character string.
\item
  \texttt{vignette()}: View a specified package vignette.
\item
  ?object\_name
\item
  ??object\_name
\item
  \texttt{rownames()}: Retrieve or set row names.
\end{itemize}

\begin{center}\rule{0.5\linewidth}{\linethickness}\end{center}

\hypertarget{loops}{%
\section{Loops}\label{loops}}

\texttt{base}

\begin{itemize}
\tightlist
\item
  \texttt{seq()}: Sequence generation.

  \begin{itemize}
  \tightlist
  \item
    This functions makes \texttt{length()} unnecessary.
  \item
    Child function: \texttt{seq\_along()}

    \begin{itemize}
    \tightlist
    \item
      In \texttt{for} loops, safer than using \texttt{ncol()} or \texttt{nrow()}.
    \end{itemize}
  \end{itemize}
\end{itemize}

\begin{center}\rule{0.5\linewidth}{\linethickness}\end{center}

\hypertarget{optimization}{%
\section{Optimization}\label{optimization}}

\texttt{microbenchmark}

\begin{itemize}
\tightlist
\item
  \texttt{microbenchmark()}: Sub-millisecond accurate timing of expression evaluations.

  \begin{itemize}
  \tightlist
  \item
    A more accurate replacement of \texttt{system.time(replicate(1000,\ expr))}.
  \end{itemize}
\end{itemize}

\begin{center}\rule{0.5\linewidth}{\linethickness}\end{center}

\hypertarget{pipes}{%
\section{Pipes}\label{pipes}}

\texttt{magrittr}

\begin{itemize}
\tightlist
\item
  \texttt{\%\textless{}\textgreater{}\%}: Compound assignment-pipe operator.
\item
  \texttt{\%\textgreater{}\%}: Forward-pipe operator.
\item
  \texttt{\%\$\%}: Expositions-pipe operator.
\item
  \texttt{add}: \texttt{+}, for pipes.
\item
  \texttt{and}: \texttt{\&}, for pipes.
\item
  \texttt{extract}:\texttt{{[}}, for pipes.
\item
  \texttt{extract2}: \texttt{{[}{[}}, for pipes.
\item
  \texttt{freduce}: Apply a list of functions sequentially.
\item
  \texttt{is\_in}: \texttt{\%in\%}, for pipes.
\item
  \texttt{multiply\_by}: \texttt{*}, for pipes.
\item
  \texttt{or}: \texttt{\textbar{}}, for pipes.
\item
  \texttt{raise\_to\_power}: \texttt{\^{}}, for pipes.
\item
  \texttt{subtract}: \texttt{-}, for pipes.
\end{itemize}

\begin{center}\rule{0.5\linewidth}{\linethickness}\end{center}

\hypertarget{popups}{%
\section{Popups}\label{popups}}

\texttt{svDialogs}

\begin{itemize}
\tightlist
\item
  \texttt{dlg\_message}: Display a modal message box (works in Windows, MacOS, and Linux).
\end{itemize}

\texttt{tcltk}

\begin{itemize}
\tightlist
\item
  \texttt{tk\_messageBox}: Display a generic message box using Tk (Windows-specific).
\end{itemize}

\hypertarget{selecting-subsetting}{%
\section{Selecting \& Subsetting}\label{selecting-subsetting}}

\begin{itemize}
\tightlist
\item
  \texttt{.\$variable\_name}

  \begin{itemize}
  \tightlist
  \item
    See example below.
  \end{itemize}
\item
  \texttt{.{[}{[}"variable\_name"{]}{]}}

  \begin{itemize}
  \tightlist
  \item
    See example below.
  \end{itemize}
\item
  \texttt{base::subset}
\item
  \texttt{dplyr::first}
\item
  \texttt{dplyr::last}
\item
  \texttt{dplyr::nth}
\item
  \texttt{dplyr::rename}
\item
  \texttt{dplyr::select}

  \begin{itemize}
  \tightlist
  \item
    Helper functions: \texttt{contains}, \texttt{ends\_with}, \texttt{matches}, \texttt{num\_range}, \texttt{one\_of}, \texttt{starts\_with}.
  \end{itemize}
\item
  `dplyr::rena
\end{itemize}

\textbf{Example:} \texttt{.\$variable\_name}

\begin{Shaded}
\begin{Highlighting}[]
\NormalTok{ui_summary_table <-}\StringTok{ }
\StringTok{  }\NormalTok{aws_vendors }\OperatorTok\StringTok{ }
\StringTok{  }\KeywordTok{filter}\NormalTok{(}\KeywordTok{str_detect}\NormalTok{(vendor_name, }\StringTok{"UTAH INTERACTIVE"}\NormalTok{)) }\OperatorTok\StringTok{ }
\StringTok{  }\NormalTok{.}\OperatorTok{$}\NormalTok{vendor_id }\OperatorTok\StringTok{ }
\StringTok{  }\KeywordTok{map}\NormalTok{(query_summary_table) }\OperatorTok\StringTok{ }
\StringTok{  }\KeywordTok{bind_rows}\NormalTok{()}
\end{Highlighting}
\end{Shaded}

\textbf{Example:} \texttt{.{[}{[}"variable\_name"{]}{]}}

\begin{Shaded}
\begin{Highlighting}[]
\NormalTok{  odbc_aws }\OperatorTok\StringTok{ }
\StringTok{    }\KeywordTok{dbGetQuery}\NormalTok{(}
      \KeywordTok{paste}\NormalTok{(}\StringTok{"}
\StringTok{            SELECT id}
\StringTok{            FROM batch}
\StringTok{            WHERE entity_id = "}\NormalTok{, t_id, }\StringTok{"}
\StringTok{            AND status IN ('PROCESSED', 'PROCESSING')"}\NormalTok{)) }\OperatorTok\StringTok{ }
\StringTok{    }\NormalTok{.[[}\StringTok{"id"}\NormalTok{]] }\OperatorTok\StringTok{ }
\StringTok{  }\KeywordTok{as.double}\NormalTok{()}
\end{Highlighting}
\end{Shaded}

\textbf{References:}

\begin{itemize}
\tightlist
\item
  \href{https://twitter.com/hadleywickham/status/643381054758363136}{``Indexing lists in \#rstats. Inspired by Residence Inn''} (Hadley Wickham, Twitter, 14 September 2015)
\end{itemize}

\begin{center}\rule{0.5\linewidth}{\linethickness}\end{center}

\hypertarget{version-control}{%
\section{Version Control}\label{version-control}}

Git

\begin{itemize}
\item
  \href{https://git-scm.com/}{Git}
\item
  \href{https://git-scm.com/book/en/v2}{\emph{Pro Git}} by Scott Chacon and Ben Straub
\item
  \href{http://r-pkgs.had.co.nz/git.html\#git-learning}{\emph{Git and GitHub}} by Hadley Wickham
\item
  \href{http://happygitwithr.com/}{\emph{Happy Git and GitHub for the useR}} by Jenny Bryan
\item
  \texttt{git\ branch}: List, create, or delete branches.

  \begin{itemize}
  \tightlist
  \item
    \texttt{git\ branch\ -d\ \textless{}branch\_name\textgreater{}}: Delete a local branch.

    \begin{itemize}
    \tightlist
    \item
      See \href{https://community.rstudio.com/t/delete-branch-in-rstudio-pop-up/15465}{``Delete branch in RStudio pop-up''} for help removing branches in RStudio after removing them from Git.
    \end{itemize}
  \end{itemize}
\end{itemize}

\texttt{\{packrat\}}

\begin{itemize}
\tightlist
\item
  See note on the \texttt{packrat} package in the ``Referenced Packages'' section.
\item
  \texttt{snapshot}: Capture and store the packages and versions in use.
\item
  \texttt{restore}: Load the most recent snapshot to the project's private library.
\end{itemize}

\hypertarget{referenced-packages}{%
\chapter{Referenced Packages}\label{referenced-packages}}

\begin{center}\rule{0.5\linewidth}{\linethickness}\end{center}

\hypertarget{a-d}{%
\section{A-D}\label{a-d}}

\href{https://CRAN.R-project.org/package=anytime}{\texttt{anytime}}: Date converter.

\href{https://CRAN.R-project.org/package=assertive}{\texttt{assertive}}: Check functions to ensure code integrity.

\href{https://www.rdocumentation.org/packages/base/versions/3.5.1}{\texttt{base}}: Base R functions.

\href{https://CRAN.R-project.org/package=bookdown}{\texttt{bookdown}}: Author books and technical documents with R Markdown.
+ See Yihui Xie's \href{https://bookdown.org/yihui/bookdown/}{\emph{bookdown: Authoring Books and Technical Documents with R Markdown}}.

\href{https://CRAN.R-project.org/package=broom}{\texttt{broom}}: Convert statistical analysis objects into tidy data frames.

\href{https://CRAN.R-project.org/package=car}{\texttt{car}}: Companion to applied regression.

\href{https://CRAN.R-project.org/package=chron}{\texttt{chron}}: Chronological objects which can handle dates and times.

\href{https://CRAN.R-project.org/package=countrycode}{\texttt{countrycode}}: Convert country names and country codes.

\href{https://CRAN.R-project.org/package=data.table}{\texttt{data.table}}: For large data.

\href{https://CRAN.R-project.org/package=DBI}{\texttt{DBI}}: Database interface.

\begin{itemize}
\item
  For MySQL documentation, see the \href{https://dev.mysql.com/}{MySQL Reference Manual}.
\item
  Use with the \texttt{odbc} package.
\end{itemize}

\href{https://CRAN.R-project.org/package=diagram}{\texttt{diagram}}: Visualize simple graphs (networks); create plot flow diagrams.

\href{https://CRAN.R-project.org/package=DiagrammeR}{\texttt{DiagrammeR}}: Graph/network visualization.
+ DiagrammeR uses the \href{https://www.graphviz.org/}{GraphViz} language.

\href{https://CRAN.R-project.org/package=dplyr}{\texttt{dplyr}}: Data manipulation.

\begin{itemize}
\tightlist
\item
  See also \href{https://dplyr.tidyverse.org/}{dplyr.tidyverse.org}.
\end{itemize}

\begin{center}\rule{0.5\linewidth}{\linethickness}\end{center}

\hypertarget{e-h}{%
\section{E-H}\label{e-h}}

\href{https://CRAN.R-project.org/package=fasttime}{\texttt{fasttime}}: Fast utilit function for time parsing and conversion.

\href{https://CRAN.R-project.org/package=forcats}{\texttt{forcats}}: Tools for working with categorical variables.

\href{https://CRAN.R-project.org/package=fuzzyjoin}{\texttt{fuzzyjoin}}: Join tables together on inexact matching.

\href{https://CRAN.R-project.org/package=fuzzywuzzyR}{\texttt{fuzzywuzzyR}}: Fuzzy string matching.

\href{https://CRAN.R-project.org/package=gdata}{\texttt{gdata}}: Data manipulation.

\href{https://CRAN.R-project.org/package=ggbeeswarm}{\texttt{ggbeeswarm}}: Categorical scatter plots (violin and beeswarm).

\href{https://CRAN.R-project.org/package=ggplot2}{\texttt{ggplot2}}: Create elegant data visualizations.

\href{https://CRAN.R-project.org/package=ggridges}{\texttt{ggridges}}: Ridgeline plots in \texttt{ggplot2}.

\href{https://www.rdocumentation.org/packages/graphics/versions/3.5.1}{\texttt{graphics}}: R functions for base graphics.

\href{https://www.rdocumentation.org/packages/grDevices/versions/3.5.2}{\texttt{grDevices}}: Grahpics devices and support for base and grid graphics.

\href{https://CRAN.R-project.org/package=Hmisc}{\texttt{Hmisc}}: Harrell miscellaneous.

\href{https://CRAN.R-project.org/package=hms}{\texttt{hms}}: Times without dates.

\href{https://CRAN.R-project.org/package=httr}{\texttt{httr}}: Tools for working with HTTP.

\begin{center}\rule{0.5\linewidth}{\linethickness}\end{center}

\hypertarget{i-l}{%
\section{I-L}\label{i-l}}

\href{https://CRAN.R-project.org/package=installr}{\texttt{installr}}: Install and update stuff (such as R, Rtools, Rstudio, Git).

\href{https://CRAN.R-project.org/package=janitor}{\texttt{janitor}}: Simple tools for examining and cleaning dirty data.

\href{https://CRAN.R-project.org/package=jsonlite}{\texttt{jsonlite}}: A robust, high performance JSON parser and generator.

\href{https://CRAN.R-project.org/package=knitr}{\texttt{knitr}}: Dynamic report generation in R using Literate Programming techniques.
+ See Yihui Xie's \href{http://yihui.name/knitr/}{\emph{knitr}}.

\href{https://CRAN.R-project.org/package=lubridate}{\texttt{lubridate}}: Functions to work with date-times and time-spans.

\begin{itemize}
\tightlist
\item
  \texttt{lubridate} uses character formatting similar to \texttt{strptime()}, though there are some differences. To see \texttt{lubridate}'s formatting, type \texttt{?parse\_date\_time} into the R Console.
\end{itemize}

\begin{center}\rule{0.5\linewidth}{\linethickness}\end{center}

\hypertarget{m-p}{%
\section{M-P}\label{m-p}}

\href{https://CRAN.R-project.org/package=magrittr}{\texttt{magrittr}}: Forward-pipe operator for R.

\href{https://www.rdocumentation.org/packages/methods/versions/3.5.1}{\texttt{methods}}: Formal methods and classes.

\href{https://CRAN.R-project.org/package=mgcv}{\texttt{mgcv}}: Mixed GAM computational vehicle with automatic smoothness estimation.

\href{https://CRAN.R-project.org/package=microbenchmark}{\texttt{microbenchmark}}: Measure and compare the execution time of R expressions.

\href{https://CRAN.R-project.org/package=odbc}{\texttt{odbc}}: Connect to ODBC compatible databases using the DBI Interface.

\href{https://CRAN.R-project.org/package=openintro}{`openintro'}: Data sets and supplemental functions from OpenIntro textbooks.

\href{https://CRAN.R-project.org/package=packrat}{\texttt{packrat}}: Manage and document the versions of packages used in an R program.

\begin{itemize}
\tightlist
\item
  \texttt{packrat} tends to cause more trouble than it prevents, so avoid using it unless necessary or until it is improved.
\end{itemize}

\href{https://CRAN.R-project.org/package=pryr}{\texttt{pryr}}: Tools to pry back the covers of R and understand the language at a deeper level.

\href{https://CRAN.R-project.org/package=purrr}{\texttt{purrr}}: Functional programming tools.

\begin{itemize}
\tightlist
\item
  See also \href{https://purrr.tidyverse.org/}{purr.tidyverse.org}.
\end{itemize}

\begin{center}\rule{0.5\linewidth}{\linethickness}\end{center}

\hypertarget{q-t}{%
\section{Q-T}\label{q-t}}

\href{https://CRAN.R-project.org/package=qdap}{\texttt{qdap}}: Bridging the gap between qualitative data and quantitative analysis (text mining).

\href{https://CRAN.R-project.org/package=qdapDictionaries}{\texttt{qdapDictionaries}}: Dictionaries and word lists for the \texttt{qdap} package.

\href{https://CRAN.R-project.org/package=RColorBrewer}{\texttt{RColorBrewer}}: ColorBrewer palattes.

\href{https://CRAN.R-project.org/package=readr}{\texttt{readr}}: Read rectangular text data.

\begin{itemize}
\tightlist
\item
  See also \href{https://readr.tidyverse.org/}{readr.tidyverse.org}.
\end{itemize}

\href{https://CRAN.R-project.org/package=readxl}{\texttt{readxl}}: Read Excel files.

\href{https://CRAN.R-project.org/package=rjson}{\texttt{rjson}}: Convert between R and JSON objects.

\href{https://CRAN.R-project.org/package=rlang}{\texttt{rlang}}: Functions for base types and Core R and Tidyverse features.

\href{https://CRAN.R-project.org/package=rmarkdown}{\texttt{RMarkdown}}: Save and execute code; generate high quality reports.

\begin{itemize}
\tightlist
\item
  See also \href{https://bookdown.org/yihui/rmarkdown/}{``R Markdown: The Definitive Guide''}).
\end{itemize}

\href{https://CRAN.R-project.org/package=scales}{\texttt{scales}}: Scale functions for visualization.

\href{https://CRAN.R-project.org/package=shiny}{\texttt{shiny}}: Web application framework.

\href{https://CRAN.R-project.org/package=splitstackshape}{\texttt{splitstackshape}}: Stack and reshape datasets after splitting concatenated values.

\href{https://www.rdocumentation.org/packages/stats/versions/3.5.1}{\texttt{stats}}: Statistical functions.

\href{https://CRAN.R-project.org/package=stringr}{\texttt{stringr}}: Working with strings.

\begin{itemize}
\tightlist
\item
  See also \href{https://stringr.tidyverse.org/}{stringr.tidyverse.org}).
\end{itemize}

\href{https://CRAN.R-project.org/package=svDialogs}{\texttt{svDialogs}}: Dialog boxes for Windows, MacOS, and Linuxes.

\href{https://www.rdocumentation.org/packages/tcltk/versions/3.5.2}{\texttt{tcltk}}: Interface and languate bindings to Tcl/Tk GUI elements.

\href{https://CRAN.R-project.org/package=tibble}{\texttt{tibble}}: Simple data frames with stricter checking and better formatting than the traditional data frame.

\href{https://CRAN.R-project.org/package=tidyr}{\texttt{tidyr}}: Tidy data.

\begin{itemize}
\tightlist
\item
  See also \href{https://tidyr.tidyverse.org/}{tidyr.tidyverse.org}.
\end{itemize}

\href{https://CRAN.R-project.org/package=tinytex}{\texttt{tinytex}}: Compile LaTeX Documents.
+ Required to compile and build a \texttt{bookdown} book.
+ See \url{https://yihui.name/tinytex/}

\begin{center}\rule{0.5\linewidth}{\linethickness}\end{center}

\hypertarget{u-z}{%
\section{U-Z}\label{u-z}}

\href{https://CRAN.R-project.org/package=R.utils}{\texttt{utils}}: Various programming utilities.

\href{https://CRAN.R-project.org/package=XLConnect}{\texttt{XLConnect}}: Read, write, and format Excel data.

\href{https://CRAN.R-project.org/package=xts}{\texttt{xts}}: Provide for uniform handling of R's different time-based data classes by extending zoo.

\href{https://CRAN.R-project.org/package=zeallot}{\texttt{zeallot}}: Multiple, unpacking, and destructuring assignment.

\href{https://CRAN.R-project.org/package=zoo}{\texttt{zoo}}: For regular and irregular time series.

\hypertarget{references-resources}{%
\chapter{References \& Resources}\label{references-resources}}

\begin{itemize}
\item
  For an introduction to the R programming language, see the R Project for Statistical Computing's \href{https://www.r-project.org/about.html}{``What is R?''} and Wikipedia's \href{https://en.wikipedia.org/wiki/R_(programming_language)}{``R (programming language).''}
\item
  To download R, go to \href{https://www.r-project.org/}{r-project.org} and choose the cloud CRAN Mirror option.
\item
  To program in the R language on a user-friendly platform, download the \href{https://www.rstudio.com/}{RStudio} IDE.
\item
  \href{https://www.r-project.org/}{The R Project for Statistical Computing}

  \begin{itemize}
  \tightlist
  \item
    \href{https://cran.r-project.org/web/packages/}{Library of R Packages}
  \item
    \href{https://www.r-project.org/help.html}{\emph{Getting Help with R}}
  \item
    \href{https://cran.r-project.org/manuals.html}{\emph{The R Manuals}}
  \item
    \href{https://cran.r-project.org/faqs.html}{\emph{Frequently Asked Questions}}
  \item
    \href{https://www.r-project.org/doc/bib/R-books.html}{\emph{Books Related to R}}
  \item
    \href{https://www.r-project.org/other-docs.html}{\emph{Documentation}}
  \end{itemize}
\item
  \href{https://www.rstudio.com/}{RStudio}

  \begin{itemize}
  \tightlist
  \item
    \href{https://www.rstudio.com/resources/cheatsheets/}{\emph{RStudio Cheat Sheets}}
  \item
    \href{https://www.rstudio.com/resources/webinars/}{\emph{Webinars and Videos On Demand}}
  \item
    \href{https://www.rstudio.com/online-learning/}{\emph{Online learning}}
  \item
    \href{https://blog.rstudio.com/}{\emph{RStudio Blog}}
  \end{itemize}
\item
  Online Manuals

  \begin{itemize}
  \tightlist
  \item
    \href{http://r4ds.had.co.nz/}{\emph{R for Data Science}}
  \item
    \href{http://adv-r.had.co.nz/}{\emph{Advanced R}} by Hadley Wickham
    + Hadley's second edition draft is available \href{https://adv-r.hadley.nz/}{here}.
  \item
    \href{http://r-pkgs.had.co.nz/}{\emph{R Packages}} by Hadley Wickham
  \item
    \href{http://www.burns-stat.com/pages/Tutor/R_inferno.pdf}{\emph{The R Inferno}} by Patrick Burns
  \item
    \href{http://style.tidyverse.org/}{\emph{The tidyverse style guide}}
  \item
    \href{https://bookdown.org/csgillespie/efficientR/}{\emph{Efficient R Programming}}
  \item
    \href{https://committedtotape.shinyapps.io/freeR/}{\emph{Free R Reading Material}}
  \end{itemize}
\item
  Other Online Resources

  \begin{itemize}
  \tightlist
  \item
    \href{https://www.datacamp.com/}{DataCamp}
  \item
    \href{https://www.rdocumentation.org/}{RDocumentation}
  \item
    \href{https://www.r-bloggers.com/how-to-learn-r-2/}{R Bloggers}

    \begin{itemize}
    \tightlist
    \item
      \href{https://www.r-bloggers.com/how-to-learn-r-2/}{``Tutorials for learning R''}
    \end{itemize}
  \item
    \href{https://regex101.com/}{Regular Expressions 101}
  \end{itemize}
\end{itemize}

\bibliography{book.bib,packages.bib}

\end{document}
