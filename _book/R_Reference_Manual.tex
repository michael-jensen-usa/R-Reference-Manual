\documentclass[]{book}
\usepackage{lmodern}
\usepackage{amssymb,amsmath}
\usepackage{ifxetex,ifluatex}
\usepackage{fixltx2e} % provides \textsubscript
\ifnum 0\ifxetex 1\fi\ifluatex 1\fi=0 % if pdftex
  \usepackage[T1]{fontenc}
  \usepackage[utf8]{inputenc}
\else % if luatex or xelatex
  \ifxetex
    \usepackage{mathspec}
  \else
    \usepackage{fontspec}
  \fi
  \defaultfontfeatures{Ligatures=TeX,Scale=MatchLowercase}
\fi
% use upquote if available, for straight quotes in verbatim environments
\IfFileExists{upquote.sty}{\usepackage{upquote}}{}
% use microtype if available
\IfFileExists{microtype.sty}{%
\usepackage{microtype}
\UseMicrotypeSet[protrusion]{basicmath} % disable protrusion for tt fonts
}{}
\usepackage[margin=1in]{geometry}
\usepackage{hyperref}
\hypersetup{unicode=true,
            pdftitle={R Reference Manual},
            pdfborder={0 0 0},
            breaklinks=true}
\urlstyle{same}  % don't use monospace font for urls
\usepackage{natbib}
\bibliographystyle{apalike}
\usepackage{color}
\usepackage{fancyvrb}
\newcommand{\VerbBar}{|}
\newcommand{\VERB}{\Verb[commandchars=\\\{\}]}
\DefineVerbatimEnvironment{Highlighting}{Verbatim}{commandchars=\\\{\}}
% Add ',fontsize=\small' for more characters per line
\usepackage{framed}
\definecolor{shadecolor}{RGB}{248,248,248}
\newenvironment{Shaded}{\begin{snugshade}}{\end{snugshade}}
\newcommand{\KeywordTok}[1]{\textcolor[rgb]{0.13,0.29,0.53}{\textbf{#1}}}
\newcommand{\DataTypeTok}[1]{\textcolor[rgb]{0.13,0.29,0.53}{#1}}
\newcommand{\DecValTok}[1]{\textcolor[rgb]{0.00,0.00,0.81}{#1}}
\newcommand{\BaseNTok}[1]{\textcolor[rgb]{0.00,0.00,0.81}{#1}}
\newcommand{\FloatTok}[1]{\textcolor[rgb]{0.00,0.00,0.81}{#1}}
\newcommand{\ConstantTok}[1]{\textcolor[rgb]{0.00,0.00,0.00}{#1}}
\newcommand{\CharTok}[1]{\textcolor[rgb]{0.31,0.60,0.02}{#1}}
\newcommand{\SpecialCharTok}[1]{\textcolor[rgb]{0.00,0.00,0.00}{#1}}
\newcommand{\StringTok}[1]{\textcolor[rgb]{0.31,0.60,0.02}{#1}}
\newcommand{\VerbatimStringTok}[1]{\textcolor[rgb]{0.31,0.60,0.02}{#1}}
\newcommand{\SpecialStringTok}[1]{\textcolor[rgb]{0.31,0.60,0.02}{#1}}
\newcommand{\ImportTok}[1]{#1}
\newcommand{\CommentTok}[1]{\textcolor[rgb]{0.56,0.35,0.01}{\textit{#1}}}
\newcommand{\DocumentationTok}[1]{\textcolor[rgb]{0.56,0.35,0.01}{\textbf{\textit{#1}}}}
\newcommand{\AnnotationTok}[1]{\textcolor[rgb]{0.56,0.35,0.01}{\textbf{\textit{#1}}}}
\newcommand{\CommentVarTok}[1]{\textcolor[rgb]{0.56,0.35,0.01}{\textbf{\textit{#1}}}}
\newcommand{\OtherTok}[1]{\textcolor[rgb]{0.56,0.35,0.01}{#1}}
\newcommand{\FunctionTok}[1]{\textcolor[rgb]{0.00,0.00,0.00}{#1}}
\newcommand{\VariableTok}[1]{\textcolor[rgb]{0.00,0.00,0.00}{#1}}
\newcommand{\ControlFlowTok}[1]{\textcolor[rgb]{0.13,0.29,0.53}{\textbf{#1}}}
\newcommand{\OperatorTok}[1]{\textcolor[rgb]{0.81,0.36,0.00}{\textbf{#1}}}
\newcommand{\BuiltInTok}[1]{#1}
\newcommand{\ExtensionTok}[1]{#1}
\newcommand{\PreprocessorTok}[1]{\textcolor[rgb]{0.56,0.35,0.01}{\textit{#1}}}
\newcommand{\AttributeTok}[1]{\textcolor[rgb]{0.77,0.63,0.00}{#1}}
\newcommand{\RegionMarkerTok}[1]{#1}
\newcommand{\InformationTok}[1]{\textcolor[rgb]{0.56,0.35,0.01}{\textbf{\textit{#1}}}}
\newcommand{\WarningTok}[1]{\textcolor[rgb]{0.56,0.35,0.01}{\textbf{\textit{#1}}}}
\newcommand{\AlertTok}[1]{\textcolor[rgb]{0.94,0.16,0.16}{#1}}
\newcommand{\ErrorTok}[1]{\textcolor[rgb]{0.64,0.00,0.00}{\textbf{#1}}}
\newcommand{\NormalTok}[1]{#1}
\usepackage{longtable,booktabs}
\usepackage{graphicx,grffile}
\makeatletter
\def\maxwidth{\ifdim\Gin@nat@width>\linewidth\linewidth\else\Gin@nat@width\fi}
\def\maxheight{\ifdim\Gin@nat@height>\textheight\textheight\else\Gin@nat@height\fi}
\makeatother
% Scale images if necessary, so that they will not overflow the page
% margins by default, and it is still possible to overwrite the defaults
% using explicit options in \includegraphics[width, height, ...]{}
\setkeys{Gin}{width=\maxwidth,height=\maxheight,keepaspectratio}
\IfFileExists{parskip.sty}{%
\usepackage{parskip}
}{% else
\setlength{\parindent}{0pt}
\setlength{\parskip}{6pt plus 2pt minus 1pt}
}
\setlength{\emergencystretch}{3em}  % prevent overfull lines
\providecommand{\tightlist}{%
  \setlength{\itemsep}{0pt}\setlength{\parskip}{0pt}}
\setcounter{secnumdepth}{5}
% Redefines (sub)paragraphs to behave more like sections
\ifx\paragraph\undefined\else
\let\oldparagraph\paragraph
\renewcommand{\paragraph}[1]{\oldparagraph{#1}\mbox{}}
\fi
\ifx\subparagraph\undefined\else
\let\oldsubparagraph\subparagraph
\renewcommand{\subparagraph}[1]{\oldsubparagraph{#1}\mbox{}}
\fi

%%% Use protect on footnotes to avoid problems with footnotes in titles
\let\rmarkdownfootnote\footnote%
\def\footnote{\protect\rmarkdownfootnote}

%%% Change title format to be more compact
\usepackage{titling}

% Create subtitle command for use in maketitle
\newcommand{\subtitle}[1]{
  \posttitle{
    \begin{center}\large#1\end{center}
    }
}

\setlength{\droptitle}{-2em}

  \title{R Reference Manual}
    \pretitle{\vspace{\droptitle}\centering\huge}
  \posttitle{\par}
    \author{}
    \preauthor{}\postauthor{}
    \date{}
    \predate{}\postdate{}
  
\usepackage{booktabs}

\usepackage{amsthm}
\newtheorem{theorem}{Theorem}[chapter]
\newtheorem{lemma}{Lemma}[chapter]
\theoremstyle{definition}
\newtheorem{definition}{Definition}[chapter]
\newtheorem{corollary}{Corollary}[chapter]
\newtheorem{proposition}{Proposition}[chapter]
\theoremstyle{definition}
\newtheorem{example}{Example}[chapter]
\theoremstyle{definition}
\newtheorem{exercise}{Exercise}[chapter]
\theoremstyle{remark}
\newtheorem*{remark}{Remark}
\newtheorem*{solution}{Solution}
\begin{document}
\maketitle

{
\setcounter{tocdepth}{1}
\tableofcontents
}
\chapter{Introduction}\label{introduction}

The purpose of this manual is to increase my ability to use R well,
which means that I must be able to get things done. My vision is that
this manual will allow me to document the material I learn in a way that
will help me work more effectively and efficiently. Most of the
documentation here will link to existing online material so that my
notes supplement rather than duplicate.

In order to help me reference material in the context I would use it, I
organized the material according to Garrett Grolemund and Hadley
Wickham's conception of the tools needed to tackle about 80\% of the
tasks required in a typical data science project
(``\href{https://r4ds.had.co.nz/introduction.html}{Introduction}'',
\emph{R for Data Science}). Those tools are: import, tidy, transform,
visualize, model, communicate, and program.

\hypertarget{htmlwidget-d25383c2133a96d5c930}{}

\chapter{Import}\label{import}

\begin{center}\rule{0.5\linewidth}{\linethickness}\end{center}

\section{Create Data}\label{create-data}

\subsection{Data Structures}\label{data-structures}

\subsubsection{Atomic Vector}\label{atomic-vector}

\texttt{base}

\begin{itemize}
\tightlist
\item
  \texttt{c()}: Combine values into a vector or list.
\item
  \texttt{factor()}: Create a factor variable.
\item
  \texttt{seq()}: Generate a sequence.
\item
  \texttt{vector()}: Create a vector.
\end{itemize}

\subsubsection{Matrix}\label{matrix}

\texttt{base}

\begin{itemize}
\tightlist
\item
  \texttt{matrix()}: Create a matrix.
\end{itemize}

\subsubsection{Array}\label{array}

\texttt{base}

\begin{itemize}
\tightlist
\item
  \texttt{array()}: Create an array.
\end{itemize}

\subsubsection{List}\label{list}

\texttt{base}

\begin{itemize}
\tightlist
\item
  \texttt{c()}: Combine values into a vector or list.
\end{itemize}

\texttt{tibble}

\begin{itemize}
\tightlist
\item
  \texttt{tibble()}: Create a data frame or list.
\end{itemize}

\subsubsection{Data Frame}\label{data-frame}

\texttt{base}

\begin{itemize}
\tightlist
\item
  \texttt{data.frame()}: Create a data frame.

  \begin{itemize}
  \tightlist
  \item
    Prefer \texttt{tibble::tibble()} over \texttt{base::data.frame()}.
  \end{itemize}
\item
  \texttt{list()}: Create a list.
\end{itemize}

\texttt{tibble}

\begin{itemize}
\tightlist
\item
  \texttt{tibble()}: Create a data frame or list.
\end{itemize}

\subsubsection{References}\label{references}

\begin{itemize}
\tightlist
\item
  \href{https://r4ds.had.co.nz/vectors.html}{``Vectors''} (Grolemund \&
  Wickham, \href{https://r4ds.had.co.nz/}{\emph{R for Data Science}})
\item
  \href{http://adv-r.had.co.nz/Data-structures.html}{``Data
  structures''} (Wickham, \href{http://adv-r.had.co.nz/}{\emph{Advanced
  R}})
\end{itemize}

\subsection{Other}\label{other}

\texttt{stats}

\begin{itemize}
\tightlist
\item
  \texttt{rnorm()}: Generate a random normal distribution.
\end{itemize}

\begin{center}\rule{0.5\linewidth}{\linethickness}\end{center}

\section{Import Data from a Local
Drive}\label{import-data-from-a-local-drive}

\texttt{base}

\begin{itemize}
\tightlist
\item
  \texttt{attach()}: Attach a set of R objects to the search path.

  \begin{itemize}
  \tightlist
  \item
    Allows objects in the database to be accessed by giving their names
    (e.g., \texttt{height} rather than \texttt{women\$height}).
  \end{itemize}
\item
  \texttt{file.choose()}: Choose a file interactively.

  \begin{itemize}
  \tightlist
  \item
    Use as \texttt{file\ =\ file.choose()} in \texttt{read.table()} and
    similar functions.
  \end{itemize}
\item
  \texttt{load()}: Reload datasets saved with \texttt{save()}.
\item
  \texttt{readRDS()}: Restore an R object written with
  \texttt{saveRDS()}.
\end{itemize}

\texttt{data.table}

\begin{itemize}
\tightlist
\item
  \texttt{fread()}: Similar to \texttt{read.table()}, but faster and
  more convenient for large data sets.
\end{itemize}

\texttt{foreign}

\begin{itemize}
\tightlist
\item
  \texttt{read.spss()}: Read an SPSS data file.
\end{itemize}

\texttt{haven}

\begin{itemize}
\tightlist
\item
  \texttt{read\_sas()}: Read and write SAS files.
\end{itemize}

\texttt{readr}

\begin{itemize}
\tightlist
\item
  \texttt{read\_delim()}: Read a delimited (including csv and txv) file.

  \begin{itemize}
  \tightlist
  \item
    Child functions: \texttt{read\_csv()}, \texttt{read\_csv2()},
    \texttt{read\_tsv()}.
  \end{itemize}
\end{itemize}

\texttt{readxl}

\begin{itemize}
\tightlist
\item
  \texttt{excel\_sheets()}: List all sheets in an Excel spreadsheet.
\end{itemize}

\texttt{utils}

\begin{itemize}
\tightlist
\item
  \texttt{data()}: Load specified data sets, or list the available data
  sets.

  \begin{itemize}
  \tightlist
  \item
    Use this function to load the data sets that accompany R packages,
    such as \texttt{openintro}'s \texttt{hsb2} and \texttt{email50} and
    \texttt{gapminder}'s \texttt{gapminder}.
  \end{itemize}
\item
  \texttt{read.table()}: Read a file in table format.

  \begin{itemize}
  \tightlist
  \item
    Child functions: \texttt{read.csv()}, \texttt{read.csv2()},
    \texttt{read.delim()}, \texttt{read.delim2()}.
  \end{itemize}
\end{itemize}

\texttt{XLConnect}

\begin{itemize}
\tightlist
\item
  \texttt{readWorksheetFromFile()}: Read data from worksheets in an
  Excel file.
\end{itemize}

\begin{center}\rule{0.5\linewidth}{\linethickness}\end{center}

\section{Import Data from the
Internet}\label{import-data-from-the-internet}

\texttt{httr}

\begin{itemize}
\tightlist
\item
  \texttt{GET()}: Get a URL.
\end{itemize}

\texttt{jsonlite}

\begin{itemize}
\tightlist
\item
  \texttt{read\_json()}: Read and write JSON.
\end{itemize}

\texttt{readr}

\begin{itemize}
\tightlist
\item
  \texttt{read\_delim()}: Read a delimited file (including csv \& tsv)
  into a tibble.

  \begin{itemize}
  \tightlist
  \item
    Child functions: \texttt{read\_csv()}, \texttt{read\_csv2()},
    \texttt{read\_tsv()}.
  \end{itemize}
\end{itemize}

\texttt{rjson}

\begin{itemize}
\tightlist
\item
  \texttt{fromJSON()}: Convert JSON to R.
\end{itemize}

\texttt{utils}

\begin{itemize}
\tightlist
\item
  \texttt{download.file()}: Download a file from the Internet.
\end{itemize}

\begin{center}\rule{0.5\linewidth}{\linethickness}\end{center}

\section{Import Data from a Database}\label{import-data-from-a-database}

\texttt{DBI}

\begin{itemize}
\tightlist
\item
  \texttt{dbBind()}: Bind values to a parameterized/prepared statement.
\item
  \texttt{dbClearResult()}: Free all resources (local and remote)
  associated with a result set.
\item
  \texttt{dbConnect()}: Connect to a DBMS.
\item
  \texttt{dbDataType()}: Determine the SQL data type of an object.
\item
  \texttt{dbDisconnect()}: Disconnect (close) a connection to a DBMS.
\item
  \texttt{dbFetch()}: Fetch records from a previously executed query.
\item
  \texttt{dbGetQuery()}: Send query, retrieve the results, and then
  clear result set.
\item
  \texttt{dbListTables()}: List remote tables.
\item
  \texttt{dbReadTable()}: Copy data frames to and from database tables.
\item
  \texttt{dbSendQuery()}: Execute a query on a given database
  connection.
\item
  \texttt{dbSendStatement()}: Execute a data manipulation statement on a
  given database connection.
\end{itemize}

\begin{center}\rule{0.5\linewidth}{\linethickness}\end{center}

\section{Reference Material}\label{reference-material}

The \texttt{openintro} package contains data sets useful for practicing
and teaching.

\chapter{Tidy}\label{tidy}

\begin{center}\rule{0.5\linewidth}{\linethickness}\end{center}

``Tidying your data means storing it in a consistent form that matches
the semantics of the dataset with the way it is stored. In brief, when
your data is tidy, each column is a variable and each row is an
observation. Tidying data is important because the consistent structure
lets you focus your struggle on questions about the data, not fighting
to get the data into the right form for different functions.''\\
- Garrett Grolemund \& Hadley Wickham, \emph{R for Data Science}

\begin{center}\rule{0.5\linewidth}{\linethickness}\end{center}

\section{Explore Raw Data}\label{explore-raw-data}

\subsection{Understand the Structure of the
Data}\label{understand-the-structure-of-the-data}

\texttt{base}

\begin{itemize}
\tightlist
\item
  \texttt{class()}: Get or set the class attribute of an object.
\item
  \texttt{colnames()}: Retrieve or set the column names of a matrix-like
  object.
\item
  \texttt{dim()}: Retrieve or set the dimension of an object.
\item
  \texttt{dimnames()}: Retrieve or set the dimension names of an object.
\item
  \texttt{format()}: Format an R object.
\item
  \texttt{length()}: Get or set the length of an object.
\item
  \texttt{levels()}: Get or set the (factor) levels of a variable.
\item
  \texttt{mode()}: Get or set the storage mode attribute of an object.

  \begin{itemize}
  \tightlist
  \item
    Modes include logical, numeric (the mode equivalent of
    \texttt{typeof()}'s integer and double), complex, character, raw,
    and list.
  \end{itemize}
\item
  \texttt{names()}: Get or set the names of an object.
\item
  \texttt{rownames()}: Retrieve or set the row names of a matrix-like
  object.
\item
  \texttt{typeof()}: Display the R internal type of an object.

  \begin{itemize}
  \tightlist
  \item
    Types include logical, integer, double, complex, character, raw, and
    list.
  \end{itemize}
\end{itemize}

\texttt{tibble}

\begin{itemize}
\tightlist
\item
  \texttt{glimpse()}: Get a glimpse of the data.

  \begin{itemize}
  \tightlist
  \item
    Similar to \texttt{utils::str()}.
  \end{itemize}
\end{itemize}

\texttt{utils}

\begin{itemize}
\tightlist
\item
  \texttt{str()}: Display the structure of an R object.

  \begin{itemize}
  \tightlist
  \item
    Similar to \texttt{tibble::glimpse()}.
  \end{itemize}
\end{itemize}

\subsection{Look at the Data}\label{look-at-the-data}

\texttt{base}

\begin{itemize}
\tightlist
\item
  \texttt{names()}: Get or set the names of an object.
\item
  \texttt{summary()}: Summarize the object.
\end{itemize}

\texttt{utils}

\begin{itemize}
\tightlist
\item
  \texttt{head()}: View the first observations in a data frame.
\item
  \texttt{tail()}: View the last observations in a data frame.
\end{itemize}

\subsection{Visualize the Data}\label{visualize-the-data}

\texttt{graphics}

\begin{itemize}
\tightlist
\item
  \texttt{hist()}: Create a histogram.
\item
  \texttt{plot()}: Create an x-y plot.
\end{itemize}

\begin{center}\rule{0.5\linewidth}{\linethickness}\end{center}

\section{Tidy Data}\label{tidy-data}

\subsection{Manage Columns and
Observations}\label{manage-columns-and-observations}

\texttt{splitstackshape}

\begin{itemize}
\tightlist
\item
  \texttt{cSplit()}: Split concatencated values into separate values.
\end{itemize}

\texttt{tidyr}

\begin{itemize}
\tightlist
\item
  \texttt{gather()}: Gather columns.
\item
  \texttt{separate()}: Separate one column into multiple columns.
\item
  \texttt{spread()}: Spread across multiple columns.
\item
  \texttt{unite()}: Unite multiple columns into one.
\end{itemize}

\subsection{Transpose}\label{transpose}

\texttt{purrr}

\begin{itemize}
\tightlist
\item
  \texttt{transpose()}: Turn a list-of-lists inside-out.
\end{itemize}

\begin{center}\rule{0.5\linewidth}{\linethickness}\end{center}

\section{Prepare Data for Analysis}\label{prepare-data-for-analysis}

\subsection{Coerce Data}\label{coerce-data}

\texttt{base}

\begin{itemize}
\tightlist
\item
  \texttt{as.*()} functions:

  \begin{itemize}
  \tightlist
  \item
    \texttt{as.array()}: Coerce to array.
  \item
    \texttt{as.data.frame()}: Coerce to data frame.

    \begin{itemize}
    \tightlist
    \item
      Prefer\texttt{tibble::as\_tibble()} to
      \texttt{base::as.data.frame()}.
    \end{itemize}
  \item
    \texttt{as.Date()}: Coerce to date.
  \item
    \texttt{as.factor()}: Coerce to factor.
  \item
    \texttt{as.list()}: Coerce to list.
  \item
    \texttt{as.matrix()}: Coerce to matrix.
  \item
    \texttt{as.POSIX*()}: Coerce to POSIXlt or POSIXct.
  \end{itemize}
\item
  \texttt{unclass()}: Remove the class attribute of an object.
\end{itemize}

\texttt{methods}

\begin{itemize}
\tightlist
\item
  \texttt{as()}: Force an object to belong to a class.
\end{itemize}

\texttt{tibble}

\begin{itemize}
\tightlist
\item
  \texttt{as\_tibble()}: Coerce lists and matrices to data frames.

  \begin{itemize}
  \tightlist
  \item
    Preferable to \texttt{base::as.data.frame()}.
  \end{itemize}
\end{itemize}

\subsection{Dates and Datetimes}\label{dates-and-datetimes}

\texttt{anytime}

\begin{itemize}
\tightlist
\item
  \texttt{anytime()}: Parse POSIXct or Date objects from input data.
\end{itemize}

\texttt{base}

\begin{itemize}
\tightlist
\item
  \texttt{as.Date()}: Date conversion to and from character.
\item
  \texttt{as.POSIX*()}: Date-time conversion for POSIXct and POSIXlt.

  \begin{itemize}
  \tightlist
  \item
    \texttt{as.POSIXct()}: Setting default for UTC and 1970.
  \end{itemize}
\item
  \texttt{strptime()}: Date-time conversion to and from character.
\item
  \texttt{Sys.timezone()}: Return the name of the current time zone.

  \begin{itemize}
  \tightlist
  \item
    \texttt{OlsonNames()} displays available time zones.
  \end{itemize}
\end{itemize}

\texttt{fasttime}

\begin{itemize}
\tightlist
\item
  \texttt{fastPOSIXct()}: Convert strings into POSICct object (string
  must be in year, month, day, hour, minute, second format.)
\end{itemize}

\texttt{hms}

\begin{itemize}
\tightlist
\item
  \texttt{hms()}: Store time-of-day values as \texttt{hms} class.

  \begin{itemize}
  \tightlist
  \item
    Child functions: \texttt{as.hms()}, `is.hms().
  \end{itemize}
\end{itemize}

\texttt{lubridate}

\begin{itemize}
\tightlist
\item
  \texttt{parse\_date\_time()}: User friendly date-time parsing
  functions that can accomodate parsing multiple dates in different
  formats.

  \begin{itemize}
  \tightlist
  \item
    \texttt{fast\_strptime()}: Fast C parser of numeric formats only
    that accepts explicit format arguments, just as
    \texttt{base::strptime()}.

    \begin{itemize}
    \tightlist
    \item
      Note that the format argument must match the input exactly,
      including any non-white space characters (such as ``T'' and
      ``Z'').
    \end{itemize}
  \item
    \texttt{make\_date()}: Create dates from numeric representations.
  \item
    \texttt{make\_datetime()}: Create date-times from numeric
    representations.
  \item
    \texttt{parse\_date\_time2()}: Fast C parser of numeric orders.
  \item
    \texttt{parse\_date\_time()} can be slow because it is designed to
    be forgiving and flexible. If the dates you are working with are in
    a consistent format (ideally ISO 8601), use one of the following:
    \texttt{fasttime::fastPOSIXct()}
  \end{itemize}
\item
  \texttt{ymd()}: Parse dates with year, month, and day components. +
  Related formats: \texttt{ydm()}, \texttt{mdy()}, \texttt{myd()},
  \texttt{dmy()}, \texttt{dym()}, \texttt{yq()}.
\item
  \texttt{ymd\_hms()}: Parse date-times with year, month, day, hour,
  minute, and second components. + Related formats: \texttt{ymd\_hm()},
  \texttt{ymd\_h()}, \texttt{dmy\_hms()}, \texttt{dmy\_hm()},
  \texttt{dmy\_h()}, \texttt{mdy\_hms()}, \texttt{mdy\_hm()},
  \texttt{mdy\_h()}, \texttt{ydm\_hms()}, \texttt{ydm\_hm()},
  \texttt{ydm\_h()}.
\end{itemize}

\subsection{Factors and Levels}\label{factors-and-levels}

\texttt{base}

\begin{itemize}
\tightlist
\item
  \texttt{factor()}: Get and set factors.

  \begin{itemize}
  \tightlist
  \item
    Rearrange the order of factors by using the \texttt{levels}
    argument. For example, rearrange the order of ``Bad,''Good," and
    ``Neutral'' using `levels = c(``Bad'', ``Neutral'', ``Good'').
  \end{itemize}
\end{itemize}

\begin{center}\rule{0.5\linewidth}{\linethickness}\end{center}

\subsection{Filter and Remove Data}\label{filter-and-remove-data}

\texttt{purrr}

\begin{itemize}
\tightlist
\item
  \texttt{keep()}: Keep or discard elements using a predicate function.
\end{itemize}

\texttt{stats}

\begin{itemize}
\tightlist
\item
  \texttt{na.omit()}: Remove rows with \texttt{NA} values.
\end{itemize}

\subsection{Strings}\label{strings}

\texttt{base}

\begin{itemize}
\tightlist
\item
  \texttt{cat()}: Concatenate and print.
\item
  \texttt{chartr()}: Change certain characters.
\item
  \texttt{grep()}: Pattern matching and replacement.

  \begin{itemize}
  \tightlist
  \item
    \texttt{grep()} family functions: \texttt{grepl()}, \texttt{sub()},
    \texttt{gsub()}, \texttt{regexpr()}, \texttt{gregexpr()}, and
    \texttt{regexec()}.
  \end{itemize}
\item
  \texttt{tolower()}: Convert to lowercase.

  \begin{itemize}
  \tightlist
  \item
    \texttt{stringr::str\_to\_lower()} is an alternative.
  \end{itemize}
\item
  \texttt{toupper()}: Convert to uppercase.

  \begin{itemize}
  \tightlist
  \item
    \texttt{stringr::str\_to\_upper()} is an alternative.
  \end{itemize}
\end{itemize}

\texttt{stringr}

\begin{itemize}
\tightlist
\item
  \texttt{str\_detect()}: Detect the presence or absence of a pattern in
  a string.

  \begin{itemize}
  \tightlist
  \item
    Control the \texttt{pattern} argument options with \texttt{regex()}
    (i.e.,
    \texttt{str\_detect(x,\ regex(pattern,\ ignore\_case\ =\ TRUE))}.
  \end{itemize}
\item
  \texttt{str\_to\_lower()}: Convert to lower case.
\item
  \texttt{str\_to\_title()}: Capitalize the first letter.
\item
  \texttt{str\_to\_upper()}: Convert to upper case.
\end{itemize}

\subsection{Test Data}\label{test-data}

\texttt{base}

\begin{itemize}
\tightlist
\item
  \texttt{all()}: Are all values true?
\item
  \texttt{any()}: Are any values true?

  \begin{itemize}
  \tightlist
  \item
    Use \texttt{any(is.na(data.frame))} to determine if there are any NA
    values in a data frame.
  \end{itemize}
\item
  \texttt{exists()}: Check whether an R object exists.
\item
  \texttt{is.*()} functions:

  \begin{itemize}
  \tightlist
  \item
    \texttt{is.array()}: Test whether an object is an array.
  \item
    \texttt{is.data.frame()}: Test whether an object is a data frame.
  \item
    \texttt{is.matrix()}: Test whether an object is a matrix.
  \item
    \texttt{is.vector()}: Test whether an object is a vector.
  \end{itemize}
\item
  \texttt{setequal()}: Check two vectors for equality.
\end{itemize}

\texttt{purrr}

\begin{itemize}
\tightlist
\item
  \texttt{every()}: Do every or some elements of a list satisfy a
  predicate?
\end{itemize}

\texttt{stats}

\begin{itemize}
\tightlist
\item
  \texttt{complete.cases()}: Find complete cases (i.e., rows without
  \texttt{NA} values).
\end{itemize}

\texttt{tibble}

\begin{itemize}
\tightlist
\item
  \texttt{is\_tibble()}: Test whether an object is a tibble.
\end{itemize}

\chapter{Transform}\label{transform}

\begin{center}\rule{0.5\linewidth}{\linethickness}\end{center}

``Transformation includes narrowing in on observations of interest (like
all people in one city, or all data from the last year), creating new
variables that are functions of existing variables (like computing
velocity from speed and time), and calculating a set of summary
statistics (like counts or means).''\\
- Garrett Grolemund \& Hadley Wickham, \emph{R for Data Science}

\begin{center}\rule{0.5\linewidth}{\linethickness}\end{center}

\section{Arithmetic \& Summary
Statistics}\label{arithmetic-summary-statistics}

\texttt{base}

\begin{itemize}
\tightlist
\item
  \texttt{abs()}: Compute absolute value.
\item
  \texttt{colMeans()}: Compute the column mean.
\item
  \texttt{colSums()}: Compute the column sum.
\item
  Comparison Operators: Binary operators which allow the comparison of
  values in atomic vectors.

  \begin{itemize}
  \tightlist
  \item
    \texttt{\textless{}}, \texttt{\textgreater{}},
    \texttt{\textless{}=}, \texttt{\textgreater{}=}, \texttt{==},
    \texttt{!=}.
  \item
    \texttt{identical()} and \texttt{all.equal()} rather than
    \texttt{==} and \texttt{!=} should be used in tests where a single
    \texttt{TRUE} or \texttt{FALSE} is required (such as \texttt{if}
    expressions).
  \end{itemize}
\item
  \texttt{diff()}: Compute the difference between two objects.
\item
  \texttt{max()}: Return the maximum value.
\item
  \texttt{mean()}: Compute the mean.
\item
  \texttt{min()}: Return the minimum value.
\item
  \texttt{round()}: Round numbers.
\item
  \texttt{rowMeans()}: Compute the row mean.
\item
  \texttt{rowSums()}: Compute the row sum.
\item
  \texttt{sqrt()}: Compute square root.
\item
  \texttt{sum()}: Sum elements.
\end{itemize}

\texttt{dplyr}

\begin{itemize}
\tightlist
\item
  \texttt{count()}: Count/tally observations by group.
\item
  \texttt{group\_by()}: Group by one or more variables.
\item
  \texttt{n()}: Get the number of observations in a current group.

  \begin{itemize}
  \tightlist
  \item
    Must be used within \texttt{summarise()}, \texttt{mutate()}, or
    \texttt{filter()}).
  \end{itemize}
\item
  \texttt{n\_distinct()}: Count the number of unique values in a vector.
\item
  \texttt{summarise()}: Reduce multiple values to a single value.
\item
  \texttt{tally()}: An alternative to \texttt{count()}.
\end{itemize}

\texttt{stats}

\begin{itemize}
\tightlist
\item
  \texttt{aggregate()}: Compute summary statistics of data subsets.
\item
  \texttt{cor()}: Correlation.
\item
  \texttt{cov()}: Covariance.
\item
  \texttt{rnorm()}: Generate a random normal distribution.
\item
  \texttt{var()}: Variance.
\end{itemize}

\begin{center}\rule{0.5\linewidth}{\linethickness}\end{center}

\section{Create New Variables or Modify Existing
Ones}\label{create-new-variables-or-modify-existing-ones}

\texttt{dplyr}

\begin{itemize}
\tightlist
\item
  \texttt{mutate()}: Add new variables.

  \begin{itemize}
  \tightlist
  \item
    \texttt{mutate()} can also be used to modify existing variables. To
    change the case of a character variable, for example, do something
    like:
  \end{itemize}
\end{itemize}

\begin{Shaded}
\begin{Highlighting}[]
\NormalTok{df <-}
\StringTok{  }\NormalTok{df }\OperatorTok
\StringTok{  }\KeywordTok{mutate}\NormalTok{(}\DataTypeTok{var_name =} \KeywordTok{str_to_lower}\NormalTok{(var_name))}
\end{Highlighting}
\end{Shaded}

\begin{center}\rule{0.5\linewidth}{\linethickness}\end{center}

\section{Dates and Datetimes}\label{dates-and-datetimes-1}

\texttt{base}

\begin{itemize}
\tightlist
\item
  \texttt{date()}: Get the current system date and time.
\item
  \texttt{difftime()}: Time intervals and differences.

  \begin{itemize}
  \tightlist
  \item
    \texttt{difftime()} is the function behind the \texttt{-} operator
    when used with dates and datetimes (e.g.,
    \texttt{time\_1\ -\ time\_2} is equivalent to
    \texttt{difftime(time\_1,\ time\_2)}). The advantage of using
    \texttt{difftime()} over \texttt{-}, however, is the \texttt{units}
    argument because it allows you to specify the unit of time in which
    the difference is calculated.
  \end{itemize}
\item
  \texttt{months()}: Extract the month names.
\item
  \texttt{quarters()}: Extract the calendar quarters.
\item
  \texttt{Sys.Date()}: Get the current date in the current time zone.
\item
  \texttt{Sys.time()}: Get the absolute date-time value (which can be
  converted to various time zones and may return different days).
\item
  \texttt{weekdays()}: Extract weekday names.
\end{itemize}

\texttt{lubridate}

\begin{itemize}
\tightlist
\item
  \texttt{date()}: Get or set the date component of a date-time.
\item
  \texttt{day()}: Get or set the day component of a datetime.
\item
  \texttt{month()}: Get or set the month component of a datetime.
\item
  \texttt{now()}: The current time (as a POSIXct object).
\item
  \texttt{quarter()}: Get or set the fiscal quarter or semester
  component of a datetime.
\item
  \texttt{round\_date()}: Round the datetime to the nearest datetime.

  \begin{itemize}
  \tightlist
  \item
    Child functions: \texttt{ceiling\_date()}, \texttt{floor\_date()}.
  \end{itemize}
\item
  Time spans: Duration:

  \begin{itemize}
  \tightlist
  \item
    \texttt{dseconds()}, \texttt{dminutes()}, \texttt{dhours()},
    \texttt{ddays()}, \texttt{dweeks()}, \texttt{dyears()}.
  \item
    Use when you are interested in seconds elapsed.
  \end{itemize}
\item
  Time spans: Interval:

  \begin{itemize}
  \tightlist
  \item
    \texttt{interval()}, \texttt{\%-\/-\%}, \texttt{is.interval()},
    \texttt{int\_start()}, \texttt{int\_end()}, \texttt{int\_length()},
    \texttt{int\_flip()}, \texttt{int\_shift()},
    \texttt{int\_overlaps()}, \texttt{int\_standardize()},
    \texttt{int\_aligns()}, \texttt{int\_diff()}.
  \item
    Use when you have a start and end.
  \end{itemize}
\item
  Time spans: Period:

  \begin{itemize}
  \tightlist
  \item
    \texttt{seconds()}, \texttt{minutes()}, \texttt{hours()},
    \texttt{days()}, \texttt{weeks()}, \texttt{months()},
    \texttt{years()}.
  \item
    Use when you are interested in human units.
  \end{itemize}
\item
  Time zones:

  \begin{itemize}
  \tightlist
  \item
    \texttt{force\_tz()}: Change the time zone without changing the
    clock time.
  \item
    \texttt{tz()}: Extract the time zone from a datetime.
  \item
    \texttt{with\_tz()}: View the same instant in a different time zone.
  \end{itemize}
\item
  \texttt{today()}: The current date (as a Date object).
\item
  \texttt{\%m+\%} \& \texttt{\%m-\%}: Add and subtract months to a date
  without exceeding the last day of the new month.
\item
  \texttt{\%within\%}: Test whether a date or interval falls within an
  interval.
\item
  \texttt{year()}: Get or set the year component of a datetime.
\end{itemize}

\begin{center}\rule{0.5\linewidth}{\linethickness}\end{center}

\section{Merge or Append Data}\label{merge-or-append-data}

\texttt{base}

\begin{itemize}
\tightlist
\item
  \texttt{append()}: Add elements to a vector.
\item
  \texttt{cbind()}: Combine objects by column.
\item
  \texttt{intersect()}: Combine data shared in common between two
  datasets.

  \begin{itemize}
  \tightlist
  \item
    Similar to \texttt{dplyr::semi\_join()}.
  \end{itemize}
\item
  \texttt{merge()}: Merge two data frames.

  \begin{itemize}
  \tightlist
  \item
    \texttt{dplyr::join} functions are an alternative to
    \texttt{merge()}.
  \end{itemize}
\item
  \texttt{rbind()}: Combine objects by row.
\item
  \texttt{setdiff()}: Find the difference between two vectors.

  \begin{itemize}
  \tightlist
  \item
    Similar to \texttt{dplyr::anti\_join()}.
  \end{itemize}
\item
  \texttt{union()}: Combine two datasets without duplicating values.
\end{itemize}

\texttt{dplyr}

\begin{itemize}
\tightlist
\item
  \texttt{bind()}: Bind multiple data frames by row and column.

  \begin{itemize}
  \tightlist
  \item
    Child functions: bind\_rows(), bind\_cols(), combine().
  \end{itemize}
\item
  Join Functions: Join two tables.

  \begin{itemize}
  \tightlist
  \item
    Filtering Joins:

    \begin{itemize}
    \tightlist
    \item
      anti\_join(): Return all rows from \texttt{x} where there are not
      matching values in \texttt{y}, keeping just columns from
      \texttt{x}.
    \item
      semi\_join(): Return all rows from \texttt{x} where there are
      matching values in \texttt{y}, keeping just columns from
      \texttt{x}. A semi join differs from an inner join because an
      inner join will return one row of \texttt{x} for each matching row
      of \texttt{y}, where a semi join will never duplicate rows of
      \texttt{x}.
    \end{itemize}
  \item
    Mutating Joins:

    \begin{itemize}
    \tightlist
    \item
      full\_join(): Return all rows and all columns from both \texttt{x}
      and \texttt{y}. Where there are not matching values, returns
      \texttt{NA} for the one missing.
    \item
      inner\_join(): Return all rows from \texttt{x} where there are
      matching values in \texttt{y}, and all columns from \texttt{x} and
      \texttt{y}. If there are multiple matches between \texttt{x} and
      \texttt{y}, all combination of the matches are returned.
    \item
      left\_join(): Return all rows from \texttt{x}, and all columns
      from \texttt{x} and \texttt{y}. Rows in \texttt{x} with no match
      in \texttt{y} will have \texttt{NA} values in the new columns. If
      there are multiple matches between \texttt{x} and \texttt{y}, all
      combinations of the matches are returned.
    \item
      right\_join(): Return all rows from \texttt{y}, and all columns
      from \texttt{x} and \texttt{y}. Rows in \texttt{x} with no match
      in \texttt{y} will have \texttt{NA} values in the new columns. If
      there are multiple matches between \texttt{y} and \texttt{x}, all
      combinations of the matches are returned.
    \end{itemize}
  \end{itemize}
\end{itemize}

\texttt{tibble}

\begin{itemize}
\tightlist
\item
  \texttt{add\_column()}: Add columns to a data frame.
\item
  \texttt{add\_row()}: Add rows to a data frame.
\end{itemize}

\begin{center}\rule{0.5\linewidth}{\linethickness}\end{center}

\section{Narrow in on Observations of
Interest}\label{narrow-in-on-observations-of-interest}

\texttt{base}

\begin{itemize}
\tightlist
\item
  \texttt{droplevels()}: Drop unused levels from factors.

  \begin{itemize}
  \tightlist
  \item
    This function will keep levels that have even 1 or 2 counts. If you
    want to remove levels with low counts from a data set in order to
    simplify your analysis, first \texttt{filter()} out those rows and
    then use \texttt{droplevels()}.
  \end{itemize}
\item
  \texttt{prop.table()}: Express table entries as proportions of the
  marginal table.

  \begin{itemize}
  \tightlist
  \item
    The input is a table produced by \texttt{table()}.
  \item
    As these are proportions of the whole,
    \texttt{sum(prop.table(table\_name))} = 1.
  \item
    Specify conditional proportions on rows or columns by using the
    \texttt{margin} argument.
  \end{itemize}
\item
  \texttt{table()}: Build a table of the counts at each combination of
  factor levels.

  \begin{itemize}
  \tightlist
  \item
    Use \texttt{prop.table()} to see the table entries expressed as
    proportions.
  \end{itemize}
\end{itemize}

\texttt{dplyr}

\begin{itemize}
\tightlist
\item
  \texttt{arrange()}: Arrange rows by variable, in ascending order.
\item
  \texttt{distinct()}: Select distinct rows.
\item
  \texttt{filter()}: Return rows with matching conditions.
\item
  \texttt{rename()}: Rename variables by name (a modification of
  \texttt{select()}).
\item
  \texttt{sample\_n()}: Sample n rows from a table.
\item
  \texttt{select()}: Select/rename variables.

  \begin{itemize}
  \tightlist
  \item
    Helper functions include: \texttt{starts\_with()},
    \texttt{ends\_with()}, \texttt{contains()}, \texttt{matches()},
    \texttt{num\_range()}, and \texttt{one\_of()}.
  \end{itemize}
\end{itemize}

\begin{center}\rule{0.5\linewidth}{\linethickness}\end{center}

\section{Test}\label{test}

\texttt{base}

\begin{itemize}
\tightlist
\item
  \texttt{setequal()}: Check two vectors for equality.
\end{itemize}

\begin{center}\rule{0.5\linewidth}{\linethickness}\end{center}

\chapter{Visualize}\label{visualize}

\begin{center}\rule{0.5\linewidth}{\linethickness}\end{center}

``Visualisation is a fundamentally human activity. A good visualisation
will show you things that you did not expect, or raise new questions
about the data. A good visualisation might also hint that you're asking
the wrong question, or you need to collect different data.
Visualisations can surprise you, but don't scale particularly well
because they require a human to interpret them.''\\
- Garrett Grolemund \& Hadley Wickham, \emph{R for Data Science}

\begin{center}\rule{0.5\linewidth}{\linethickness}\end{center}

\section{Flowcharts}\label{flowcharts}

\texttt{diagram}

\texttt{DiagrammeR}

\begin{center}\rule{0.5\linewidth}{\linethickness}\end{center}

\section{Charts}\label{charts}

\texttt{ggplot2}

\begin{itemize}
\tightlist
\item
  \texttt{ggplot()}: Create a plot.
\item
  \texttt{facet\_wrap()}: Wrap a 1D ribbon of panels into 2D (observe a
  variable, conditional on another variable).
\end{itemize}

\texttt{graphics}

\begin{itemize}
\tightlist
\item
  \texttt{boxplot()}: Create a box-and-whisker plot.
\end{itemize}

\begin{center}\rule{0.5\linewidth}{\linethickness}\end{center}

\section{Interfaces}\label{interfaces}

\texttt{shiny}

\chapter{Model}\label{model}

\begin{center}\rule{0.5\linewidth}{\linethickness}\end{center}

``Models are complementary tools to visualisation. Once you have made
your questions sufficiently precise, you can use a model to answer them.
Models are a fundamentally mathematical or computational tool, so they
generally scale well. \ldots{} But every model makes assumptions, and by
its very nature a model cannot question its own assumptions. That means
a model cannot fundamentally surprise you. - Garrett Grolemund \& Hadley
Wickham, \emph{R for Data Science}''

\begin{center}\rule{0.5\linewidth}{\linethickness}\end{center}

\section{}\label{section}

\texttt{broom}

\begin{itemize}
\tightlist
\item
  \texttt{tidy()}: Construct a data frame that summarizes the model's
  statistical findings.
\end{itemize}

\texttt{dplyr}

\begin{itemize}
\tightlist
\item
  \texttt{sample\_n()}: Sample n rows from a table.
\end{itemize}

\texttt{stats}: Statistical functions.

\begin{itemize}
\tightlist
\item
  \texttt{coef()}: Extract model coefficients.
\end{itemize}

\chapter{Communicate}\label{communicate}

\begin{center}\rule{0.5\linewidth}{\linethickness}\end{center}

``The last step of data science is communication, an absolutely critical
part of any data analysis project. It doesn't matter how well your
models and visualisation have led you to understand the data unless you
can also communicate your results to others.''\\
- Garrett Grolemund \& Hadley Wickham, \emph{R for Data Science}

\begin{center}\rule{0.5\linewidth}{\linethickness}\end{center}

\section{Format Output}\label{format-output}

\texttt{base}

\begin{itemize}
\tightlist
\item
  \texttt{format()}: Format an object for pretty printing.
\end{itemize}

\texttt{lubridate}

\begin{itemize}
\tightlist
\item
  \texttt{stamp()}: Format dates and times based on human-friendly
  templates.
\end{itemize}

\texttt{scales}: Scale functions for visualization.

\begin{itemize}
\tightlist
\item
  \texttt{dollar()}: Round to the nearest cent and display dollar sign.
\end{itemize}

\begin{center}\rule{0.5\linewidth}{\linethickness}\end{center}

\section{Export}\label{export}

\texttt{base}

\begin{itemize}
\tightlist
\item
  \texttt{file.path()}: Construct a file path.
\item
  \texttt{save()}: Save R objects.
\item
  \texttt{saveRDS()}: Save a single R object.

  \begin{itemize}
  \tightlist
  \item
    See
    \href{https://www.fromthebottomoftheheap.net/2012/04/01/saving-and-loading-r-objects/}{``A
    better way of saving and loading objects in R''} to understand the
    differences between \texttt{save()} and \texttt{saveRDS()}.
  \end{itemize}
\end{itemize}

\texttt{readr}

\begin{itemize}
\tightlist
\item
  \texttt{write\_delim()}: Write a data frame to a delimited file.

  \begin{itemize}
  \tightlist
  \item
    About twice as fast as \texttt{write.csv()} and never writes row
    names.
  \item
    Child functions: \texttt{write\_csv()},
    \texttt{write\_excel\_csv()}, \texttt{write\_tsv()}.
  \end{itemize}
\end{itemize}

\texttt{utils}

\begin{itemize}
\tightlist
\item
  \texttt{write.table()}: Data output.

  \begin{itemize}
  \tightlist
  \item
    Prefer \texttt{readr::write\_delim()} to
    \texttt{utils::write.table()}.
  \item
    Child functions: \texttt{write.csv()}, \texttt{write.csv2()}.
  \end{itemize}
\end{itemize}

\texttt{XLConnect}: Read, write, and format Excel data.

\chapter{Program}\label{program}

\begin{center}\rule{0.5\linewidth}{\linethickness}\end{center}

``Surrounding {[}the tools for importing, tidying, transforming,
visualising, modeling, and communicating data{]} is programming.
Programming is a cross-cutting tool that you use in every part of a
project. You don't need to be an expert programmer to be a data
scientist, but learning more about programming pays off because becoming
a better programmer allows you to automate common tasks, and solve new
problems with greater ease.''\\
- Garrett Grolemund \& Hadley Wickham, \emph{R for Data Science}

\begin{center}\rule{0.5\linewidth}{\linethickness}\end{center}

\section{Conditionals}\label{conditionals}

\texttt{base}

\begin{itemize}
\tightlist
\item
  'ifelse()`: Conditional element selection.
\end{itemize}

\texttt{dplyr}

\begin{itemize}
\tightlist
\item
  \texttt{case\_when()}: A general vectorized if.
\end{itemize}

\begin{center}\rule{0.5\linewidth}{\linethickness}\end{center}

\section{Environment and Workspace}\label{environment-and-workspace}

\texttt{base}

\begin{itemize}
\tightlist
\item
  \texttt{dir()}: List the files in a directory/folder.
\item
  Environments

  \begin{itemize}
  \tightlist
  \item
    \texttt{baseenv()}: The environment of the \texttt{base} package,
    it's enclosing environment (``parent environment'') is the empty
    environment.
  \item
    \texttt{emptyenv()}: The empty environment, which is the ancestor of
    all environments and the only environment without an enclosing
    environment.
  \item
    \texttt{environment()}: The current environment.
  \item
    \texttt{globalenv()}: The environment in which you normally work,
    it's enclosing environment is the last package attached with
    \texttt{library()} or \texttt{require()}.
  \item
    \texttt{new.env()}: Create a new environment.
  \end{itemize}
\item
  \texttt{exists()}: Look for an R object of the given name and possibly
  return it.

  \begin{itemize}
  \tightlist
  \item
    Must use quotations to name the object.
  \end{itemize}
\item
  \texttt{getwd()}: Get the working directory.\\
\item
  \texttt{list.files()}: List the files in a directory/folder.
\item
  \texttt{ls()}: List objects in the specified environment.
\item
  \texttt{options()}: Set and examine global options.

  \begin{itemize}
  \tightlist
  \item
    \texttt{getOption()}: Set and examine global options.
  \end{itemize}
\item
  \texttt{rm()}: Remove objects from a specified environment.
\item
  \texttt{search()}: Return a list of attached packages and R objects.

  \begin{itemize}
  \tightlist
  \item
    \texttt{searchpaths()}: Return the path to attached packages.
  \end{itemize}
\end{itemize}

\texttt{gdata}

\begin{itemize}
\tightlist
\item
  \texttt{object.size()}: Report the space allocated for an object.

  \begin{itemize}
  \tightlist
  \item
    See also \texttt{utils::object.size()}.
  \end{itemize}
\end{itemize}

\texttt{installr}

\begin{itemize}
\tightlist
\item
  \texttt{updateR()}: Check for the latest R version; downloads and
  installs new R versions.
\end{itemize}

\texttt{pryr}

\begin{itemize}
\tightlist
\item
  \texttt{where()}: Find where a name is defined.
\end{itemize}

\texttt{utils}

\begin{itemize}
\tightlist
\item
  \texttt{object.size()}: Report the space allocated for an object.

  \begin{itemize}
  \tightlist
  \item
    See also \texttt{gdata::object.size()}.
  \end{itemize}
\end{itemize}

References:

\begin{itemize}
\tightlist
\item
  \href{http://adv-r.had.co.nz/Environments.html\#environments}{``Environments''}
  (Hadley Wickham, \href{http://adv-r.had.co.nz/}{\emph{Advanced R}})
\end{itemize}

\begin{center}\rule{0.5\linewidth}{\linethickness}\end{center}

\section{Evaluation (Standard and
Non-standard)}\label{evaluation-standard-and-non-standard}

\texttt{base}

\begin{itemize}
\tightlist
\item
  \texttt{cat()}: Concatenate and print.
\item
  \texttt{quote()}: Return the argument, unevaluated.
\item
  \texttt{writeLines()}: Display quotes and backslashes as they would be
  read, rather than as R stores them (i.e., see the raw contents of the
  string, as the \texttt{print()} representation is not the same as the
  string itself).
\end{itemize}

\texttt{rlang}

\begin{itemize}
\tightlist
\item
  Quosures

  \begin{itemize}
  \tightlist
  \item
    \texttt{enquo()}, \texttt{new\_quosure()}, \texttt{quo()}.
  \end{itemize}
\end{itemize}

References:

\begin{itemize}
\tightlist
\item
  \href{http://adv-r.had.co.nz/Computing-on-the-language.html}{``Non-standard
  evaluation''} (Hadley Wickham,
  \href{http://adv-r.had.co.nz/}{\emph{Advanced R}})
\item
  \href{https://cran.r-project.org/web/packages/lazyeval/vignettes/lazyeval.html}{``Non-standard
  evaluation''} (Hadley Wickham, \texttt{lazyeval} package vignette)
\item
  \href{https://dplyr.tidyverse.org/articles/programming.html}{``Programming
  with dplyr''} (dplyr.tidyverse.org)
\end{itemize}

\begin{center}\rule{0.5\linewidth}{\linethickness}\end{center}

\section{Functionals}\label{functionals}

\texttt{base}

\begin{itemize}
\tightlist
\item
  Apply Functions

  \begin{itemize}
  \tightlist
  \item
    \texttt{apply()}: Apply functions over array margins.
  \item
    \texttt{lapply()}: Apply a function over a list or vector.
  \item
    \texttt{sapply()}: Apply a function over a list or vector and return
    a vector or matrix.
  \item
    \texttt{vapply()}: A safer version of \texttt{sapply()}, as it
    requires the output type to be predetermined.
  \item
    \texttt{mapply()}: Apply a function to multiple list or vector
    arguments.
  \item
    \texttt{rapply()}: Recursively apply a function to a list.
  \item
    \texttt{tapply()}: Apply a function over a ragged array.
  \end{itemize}
\end{itemize}

\texttt{purrr}

\begin{itemize}
\tightlist
\item
  \texttt{map()}: Apply a function to each element of a vector.
\item
  \texttt{map2()}: Map over multiple inputs simultaneously.
\item
  \texttt{safely()}: Capture side effects.
\item
  \texttt{transpose()}: Transpose a list (turn a list-of-lists
  inside-out).
\end{itemize}

\begin{center}\rule{0.5\linewidth}{\linethickness}\end{center}

\section{Functions}\label{functions}

\texttt{base}

\begin{itemize}
\tightlist
\item
  \texttt{do.call()}): Execute a function call from a name or a function
  and a list of arguments to be passed to the function.
\item
  \texttt{message()}: Generage a diagnostic message.
\item
  \texttt{unlist()}: Flatten lists.

  \begin{itemize}
  \tightlist
  \item
    Useful when using \texttt{purrr}'s \texttt{map()} functions, which
    return objects as type \texttt{list}.
  \end{itemize}
\end{itemize}

\begin{center}\rule{0.5\linewidth}{\linethickness}\end{center}

\section{Learn About an Object}\label{learn-about-an-object}

\texttt{base}

\begin{itemize}
\tightlist
\item
  \texttt{args()}: Display the argument names and default values of a
  function.
\item
  \texttt{attributes()}: View or assign an objects attributes (e.g.,
  \texttt{class()}, \texttt{dim()}, \texttt{dimnames()},
  \texttt{names()}, \texttt{row.names()}).
\item
  \texttt{body()}: Get or set the body of a function.
\item
  \texttt{colnames()}: Retrieve or set column names.
\item
  \texttt{dim()}: Retrieve or set the dimnames of an object.
\item
  \texttt{dimnames()}: Retrieve or set the dimension names of an object.
\item
  \texttt{formals()}: Get or set the formal arguments of a function.
\item
  \texttt{help()}: Get the topic documentation.
\item
  \texttt{vignette()}: View a specified package vignette.
\item
  ?object\_name
\item
  ??object\_name
\item
  \texttt{rownames()}: Retrieve or set row names.
\end{itemize}

\begin{center}\rule{0.5\linewidth}{\linethickness}\end{center}

\section{Optimization}\label{optimization}

\texttt{microbenchmark}

\begin{itemize}
\tightlist
\item
  \texttt{microbenchmark()}: Sub-millisecond accurate timing of
  expression evaluations.

  \begin{itemize}
  \tightlist
  \item
    A more accurate replacement of
    \texttt{system.time(replicate(1000,\ expr))}.
  \end{itemize}
\end{itemize}

\begin{center}\rule{0.5\linewidth}{\linethickness}\end{center}

\section{Pipes}\label{pipes}

\texttt{magrittr}: Forward-pipe operator for R.

\begin{itemize}
\tightlist
\item
  \texttt{freduce()}: Apply a list of functions sequentially.
\item
  \texttt{\%\textless{}\textgreater{}\%}: Compound assignment-pipe
  operator.
\item
  \texttt{\%\textgreater{}\%}: Forward-pipe operator.
\item
  \texttt{\%\$\%}: Expositions-pipe operator.
\end{itemize}

\begin{center}\rule{0.5\linewidth}{\linethickness}\end{center}

\section{Selecting \& Subsetting}\label{selecting-subsetting}

\texttt{base}

\begin{itemize}
\tightlist
\item
  \texttt{subset()}: Subset vectors, matrices, and data frames.
\end{itemize}

\texttt{dplyr}

\begin{itemize}
\tightlist
\item
  \texttt{first()}: Extract the first element of a vector.
\item
  \texttt{last()}: Extract the last element of a vector.
\item
  `nth(): Extract the nth element of a vector.
\item
  \texttt{select()}: Select/rename variables.

  \begin{itemize}
  \tightlist
  \item
    Helper functions include: \texttt{starts\_with()},
    \texttt{ends\_with()}, \texttt{contains()}, \texttt{matches()},
    \texttt{num\_range()}, and \texttt{one\_of()}.
  \item
    A closely-related function is \texttt{dplyr::rename()}.
  \end{itemize}
\end{itemize}

References:

\begin{itemize}
\tightlist
\item
  \href{https://twitter.com/hadleywickham/status/643381054758363136}{``Indexing
  lists in \#rstats. Inspired by Residence Inn''} (Hadley Wickham,
  Twitter, 14 September 2015)
\end{itemize}

\begin{center}\rule{0.5\linewidth}{\linethickness}\end{center}

\section{Version Control}\label{version-control}

Git

\begin{itemize}
\tightlist
\item
  \href{https://git-scm.com/}{Git}
\item
  \href{https://git-scm.com/book/en/v2}{\emph{Pro Git}} by Scott Chacon
  and Ben Straub
\item
  \href{http://r-pkgs.had.co.nz/git.html\#git-learning}{\emph{Git and
  GitHub}} by Hadley Wickham
\item
  \href{http://happygitwithr.com/}{\emph{Happy Git and GitHub for the
  useR}} by Jenny Bryan
\end{itemize}

\texttt{packrat}

\begin{itemize}
\tightlist
\item
  See note on the \texttt{packrat} package in the ``Referenced
  Packages'' section.
\item
  \texttt{snapshot()}: Capture and store the packages and versions in
  use.
\item
  \texttt{restore()}: Load the most recent snapshot to the project's
  private library.
\end{itemize}

\chapter{Referenced Packages}\label{referenced-packages}

\begin{center}\rule{0.5\linewidth}{\linethickness}\end{center}

\section{A-D}\label{a-d}

\href{https://CRAN.R-project.org/package=anytime}{\texttt{anytime}}:
Date converter.

\href{https://www.rdocumentation.org/packages/base/versions/3.5.1}{\texttt{base}}:
Base R functions.

\href{https://CRAN.R-project.org/package=bookdown}{\texttt{bookdown}}:
Author books and technical documents with R Markdown. + See Yihui Xie's
\href{https://bookdown.org/yihui/bookdown/}{\emph{bookdown: Authoring
Books and Technical Documents with R Markdown}}.

\href{https://CRAN.R-project.org/package=broom}{\texttt{broom}}: Convert
statistical analysis objects into tidy data frames.

\href{https://CRAN.R-project.org/package=chron}{\texttt{chron}}:
Chronological objects which can handle dates and times.

\href{https://CRAN.R-project.org/package=data.table}{\texttt{data.table}}:
For large data.

\href{https://CRAN.R-project.org/package=DBI}{\texttt{DBI}}: Database
interface.

\begin{itemize}
\item
  For MySQL documentation, see the \href{https://dev.mysql.com/}{MySQL
  Reference Manual}.
\item
  Use with the \texttt{odbc} package.
\end{itemize}

\href{https://CRAN.R-project.org/package=diagram}{\texttt{diagram}}:
Visualize simple graphs (networks); create plot flow diagrams.

\href{https://CRAN.R-project.org/package=DiagrammeR}{\texttt{DiagrammeR}}:
Graph/network visualization. + DiagrammeR uses the
\href{https://www.graphviz.org/}{GraphViz} language.

\href{https://CRAN.R-project.org/package=dplyr}{\texttt{dplyr}}: Data
manipulation.

\begin{itemize}
\tightlist
\item
  See also \href{https://dplyr.tidyverse.org/}{dplyr.tidyverse.org}.
\end{itemize}

\begin{center}\rule{0.5\linewidth}{\linethickness}\end{center}

\section{E-H}\label{e-h}

\href{https://CRAN.R-project.org/package=fasttime}{\texttt{fasttime}}:
Fast utilit function for time parsing and conversion.

\href{https://CRAN.R-project.org/package=forcats}{\texttt{forcats}}:
Tools for working with categorical variables.

\href{https://CRAN.R-project.org/package=gdata}{\texttt{gdata}}: Data
manipulation.

\href{https://CRAN.R-project.org/package=ggplot2}{\texttt{ggplot2}}:
Create elegant data visualizations.

\href{https://www.rdocumentation.org/packages/graphics/versions/3.5.1}{\texttt{graphics}}:
R functions for base graphics.

\href{https://CRAN.R-project.org/package=hms}{\texttt{hms}}: Times
without dates.

\href{https://CRAN.R-project.org/package=httr}{\texttt{httr}}: Tools for
working with HTTP.

\begin{center}\rule{0.5\linewidth}{\linethickness}\end{center}

\section{I-L}\label{i-l}

\href{https://CRAN.R-project.org/package=installr}{\texttt{installr}}:
Install and update stuff (such as R, Rtools, Rstudio, Git).

\href{https://CRAN.R-project.org/package=jsonlite}{\texttt{jsonlite}}: A
robust, high performance JSON parser and generator.

\href{https://CRAN.R-project.org/package=knitr}{\texttt{knitr}}: Dynamic
report generation in R using Literate Programming techniques. + See
Yihui Xie's \href{http://yihui.name/knitr/}{\emph{knitr}}.

\href{https://CRAN.R-project.org/package=lubridate}{\texttt{lubridate}}:
Functions to work with date-times and time-spans.

\begin{itemize}
\tightlist
\item
  \texttt{lubridate} uses character formatting similar to
  \texttt{strptime()}, though there are some differences. To see
  \texttt{lubridate}'s formatting, type \texttt{?parse\_date\_time} into
  the R Console.
\end{itemize}

\begin{center}\rule{0.5\linewidth}{\linethickness}\end{center}

\section{M-P}\label{m-p}

\href{https://CRAN.R-project.org/package=magrittr}{\texttt{magrittr}}:
Forward-pipe operator for R.

\href{https://www.rdocumentation.org/packages/methods/versions/3.5.1}{\texttt{methods}}:
Formal methods and classes.

\href{https://CRAN.R-project.org/package=microbenchmark}{\texttt{microbenchmark}}:
Measure and compare the execution time of R expressions.

\href{https://CRAN.R-project.org/package=odbc}{\texttt{odbc}}: Connect
to ODBC compatible databases using the DBI Interface.

\href{https://CRAN.R-project.org/package=packrat}{\texttt{packrat}}:
Manage and document the versions of packages used in an R program.

\begin{itemize}
\tightlist
\item
  \texttt{packrat} tends to cause more trouble than it prevents, so
  avoid using it unless necessary or until it is improved.
\end{itemize}

\href{https://CRAN.R-project.org/package=pryr}{\texttt{pryr}}: Tools to
pry back the covers of R and understand the language at a deeper level.

\href{https://CRAN.R-project.org/package=purrr}{\texttt{purrr}}:
Functional programming tools.

\begin{itemize}
\tightlist
\item
  See also \href{https://purrr.tidyverse.org/}{purr.tidyverse.org}.
\end{itemize}

\begin{center}\rule{0.5\linewidth}{\linethickness}\end{center}

\section{Q-T}\label{q-t}

\href{https://CRAN.R-project.org/package=readr}{\texttt{readr}}: Read
rectangular text data.

\begin{itemize}
\tightlist
\item
  See also \href{https://readr.tidyverse.org/}{readr.tidyverse.org}.
\end{itemize}

\href{https://CRAN.R-project.org/package=readxl}{\texttt{readxl}}: Read
Excel files.

\href{https://CRAN.R-project.org/package=rjson}{\texttt{rjson}}: Convert
between R and JSON objects.

\href{https://CRAN.R-project.org/package=rlang}{\texttt{rlang}}:
Functions for base types and Core R and Tidyverse features.

\href{https://CRAN.R-project.org/package=rmarkdown}{\texttt{RMarkdown}}:
Save and execute code; generate high quality reports.

\begin{itemize}
\tightlist
\item
  See also \href{https://bookdown.org/yihui/rmarkdown/}{``R Markdown:
  The Definitive Guide''}).
\end{itemize}

\href{https://CRAN.R-project.org/package=scales}{\texttt{scales}}: Scale
functions for visualization.

\href{https://CRAN.R-project.org/package=shiny}{\texttt{shiny}}: Web
application framework.

\href{https://CRAN.R-project.org/package=splitstackshape}{\texttt{splitstackshape}}:
Stack and reshape datasets after splitting concatenated values.

\href{https://www.rdocumentation.org/packages/stats/versions/3.5.1}{\texttt{stats}}:
Statistical functions.

\href{https://CRAN.R-project.org/package=stringr}{\texttt{stringr}}:
Working with strings.

\begin{itemize}
\tightlist
\item
  See also
  \href{https://stringr.tidyverse.org/}{stringr.tidyverse.org}).
\end{itemize}

\href{https://CRAN.R-project.org/package=tibble}{\texttt{tibble}}:
Simple data frames with stricter checking and better formatting than the
traditional data frame.

\href{https://CRAN.R-project.org/package=tidyr}{\texttt{tidyr}}: Tidy
data.

\begin{itemize}
\tightlist
\item
  See also \href{https://tidyr.tidyverse.org/}{tidyr.tidyverse.org}.
\end{itemize}

\href{https://CRAN.R-project.org/package=tinytex}{\texttt{tinytex}}:
Compile LaTeX Documents. + Required to compile and build a
\texttt{bookdown} book. + See \url{https://yihui.name/tinytex/}

\begin{center}\rule{0.5\linewidth}{\linethickness}\end{center}

\section{U-Z}\label{u-z}

\href{https://CRAN.R-project.org/package=R.utils}{\texttt{utils}}:
Various programming utilities.

\href{https://CRAN.R-project.org/package=XLConnect}{\texttt{XLConnect}}:
Read, write, and format Excel data.

\href{https://CRAN.R-project.org/package=xts}{\texttt{xts}}: Provide for
uniform handling of R's different time-based data classes by extending
zoo.

\href{https://CRAN.R-project.org/package=zoo}{\texttt{zoo}}: For regular
and irregular time series.

\chapter{References \& Resources}\label{references-resources}

\begin{itemize}
\item
  For an introduction to the R programming language, see the R Project
  for Statistical Computing's
  \href{https://www.r-project.org/about.html}{``What is R?''} and
  Wikipedia's
  \href{https://en.wikipedia.org/wiki/R_(programming_language)}{``R
  (programming language).''}
\item
  To download R, go to \href{https://www.r-project.org/}{r-project.org}
  and choose the cloud CRAN Mirror option.
\item
  To program in the R language on a user-friendly platform, download the
  \href{https://www.rstudio.com/}{RStudio} IDE.
\item
  \href{https://www.r-project.org/}{The R Project for Statistical
  Computing}

  \begin{itemize}
  \tightlist
  \item
    \href{https://cran.r-project.org/web/packages/}{Library of R
    Packages}
  \item
    \href{https://www.r-project.org/help.html}{\emph{Getting Help with
    R}}
  \item
    \href{https://cran.r-project.org/manuals.html}{\emph{The R Manuals}}
  \item
    \href{https://cran.r-project.org/faqs.html}{\emph{Frequently Asked
    Questions}}
  \item
    \href{https://www.r-project.org/doc/bib/R-books.html}{\emph{Books
    Related to R}}
  \item
    \href{https://www.r-project.org/other-docs.html}{\emph{Documentation}}
  \end{itemize}
\item
  \href{https://www.rstudio.com/}{RStudio}

  \begin{itemize}
  \tightlist
  \item
    \href{https://www.rstudio.com/resources/cheatsheets/}{\emph{RStudio
    Cheat Sheets}}
  \item
    \href{https://www.rstudio.com/resources/webinars/}{\emph{Webinars
    and Videos On Demand}}
  \item
    \href{https://www.rstudio.com/online-learning/}{\emph{Online
    learning}}
  \item
    \href{https://blog.rstudio.com/}{\emph{RStudio Blog}}
  \end{itemize}
\item
  Online Manuals

  \begin{itemize}
  \tightlist
  \item
    \href{http://r4ds.had.co.nz/}{\emph{R for Data Science}}
  \item
    \href{http://adv-r.had.co.nz/}{\emph{Advanced R}} by Hadley Wickham

    \begin{itemize}
    \tightlist
    \item
      Hadley's second edition draft is available
      \href{https://adv-r.hadley.nz/}{here}.
    \end{itemize}
  \item
    \href{http://r-pkgs.had.co.nz/}{\emph{R Packages}} by Hadley Wickham
  \item
    \href{http://style.tidyverse.org/}{\emph{The tidyverse style guide}}
  \item
    \href{https://bookdown.org/csgillespie/efficientR/}{\emph{Efficient
    R Programming}}
  \end{itemize}
\item
  Other Online Resources

  \begin{itemize}
  \tightlist
  \item
    \href{https://www.datacamp.com/}{DataCamp}
  \item
    \href{https://www.rdocumentation.org/}{RDocumentation}
  \item
    \href{https://www.r-bloggers.com/how-to-learn-r-2/}{R Bloggers}

    \begin{itemize}
    \tightlist
    \item
      \href{https://www.r-bloggers.com/how-to-learn-r-2/}{``Tutorials
      for learning R''}
    \end{itemize}
  \end{itemize}
\end{itemize}

\bibliography{book.bib,packages.bib}


\end{document}
